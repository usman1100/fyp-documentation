\documentclass{article}
\usepackage[english]{babel}
\usepackage{blindtext}
\usepackage[hidelinks]{hyperref}
\usepackage[margin=0.5in]{geometry}
\usepackage{fancyhdr}
\usepackage{graphicx}
\usepackage{float}
\usepackage{booktabs}
\usepackage{titlesec}
\usepackage{tabularx}
\usepackage[shortlabels]{enumitem}

% \renewcommand\headrulewidth{0pt} % remove header rule
\fancyhf{}
\fancyhead[L]{\vspace{3mm}Session 2018--2022}
\fancyhead[R]{Software Cost and Effort Estimation System}
\fancyfoot[C]{\thepage}
\pagestyle{fancy}

\graphicspath{ {./} }

% Code to use \paragraph as \subsubsubsection ----

\makeatletter
\renewcommand\paragraph{\@startsection{paragraph}{4}{\z@}%
            {-2.5ex\@plus -1ex \@minus -.25ex}%
            {1.25ex \@plus .25ex}%
            {\normalfont\normalsize\bfseries}}
\makeatother
\setcounter{secnumdepth}{4} % how many sectioning levels to assign numbers to
\setcounter{tocdepth}{4}    % how many sectioning levels to show in ToC
% ---------------------------------------------------- 



\begin{document}


\pagenumbering{roman}

% Title Page
\graphicspath{ {./images/} }



\begin{center}
	\vspace{10mm}

	\includegraphics[height=3.5cm, width=10cm]{logo} \\
	\vspace{10mm}

	\Huge{\textbf{Software Cost and Effort Estimation System}} \\
	\huge{A tool for software industries}
	\vspace{10mm}

	\Huge{\textbf{Final Year Project Documentation}} \\
	\vspace{20mm}

	\Large{\textbf{Submitted by}} \\
	\vspace{15mm}

	\Large{\textbf{Usman Ahmed \hspace{3cm} 70066997}}

	\vspace{1mm}

	\Large{\textbf{Faizan Ahmed \hspace{3cm} 70068241}}

	\vspace{10mm}

	\Large{\textbf{Project Supervisor}} \\
	\vspace{3mm}
	\Large{\textbf{Dr. Yasir Mehmood}}

	\vspace{15mm}


	\Large{\textbf{BACHELOR OF SCIENCE IN SOFTWARE ENGINEERING}} \\
	\vspace{10mm}
	\Large{\textbf{DEPARTMENT OF SOFTWARE ENGINEERING}} \\
	\vspace{3mm}
	\Large{\textbf{THE UNIVERSITY OF LAHORE}} \\






\end{center}


\pagebreak

% Signatures Page


\vspace{50mm}
\begin{center}

	
	\huge{\textbf{FINAL YEAR PROJECT PHASE-I DOCUMENTATION}} \\
	\vspace{10mm}

\end{center}

\vspace{20mm}

\begin{center}
	\huge{\textbf{STATEMENT OF SUBMISSION}} \\
	\vspace{10mm}

\end{center}


\vspace{20mm}


\begin{center}
	\Large{Submitted to the University of Lahore in partial fulfillment of the requirement for the award of degree of Bachelors of Science in Software Engineering (BSSE)} \\
	\vspace{10mm}

\end{center}
\vspace{10mm}
\Large

By
\vspace{10mm}

Usman Ahmed \hspace{5mm} (70066997) \hspace{5mm} \hrulefill

\vspace{5mm}

Faizan Ahmed \hspace{6mm} (70068241) \hspace{5mm} \hrulefill

\vspace{15mm}

\textbf{Project Advisor}
\hrulefill
\vspace{5mm}



\textbf{Dr Yasir Mahmood}
\vspace{5mm}


(Assistant Professor UOL)
\vspace{5mm}


Department of Software Engineering



\pagebreak


% Abstract, Dedication and Acknowledgements
\section*{Abstract}
\addcontentsline{toc}{section}{Abstract}
Cost and effort estimation is a critical part of any major project. A wrong estimate can lead to a catastrophic failure. According to a survey, over 60\% of all software projects fail to complete within their estimated cost and time budget. On top of that, our methods for calculating the cost and effort of a software project are not very accurate. Most of the organizations make project managers calculate the current project's cost and effort by comparing it to a similar project they have completed before. This method is flawed since it assumes that the features of the project are excatly the same as the features of the previous projects. This is obviously not true as different project have different features that are driven from their scope, industry, and other factors. We aspire to develop system that can introduce new and far better solutions to the problem of cost and effort estimation. Our system will provide many ways of estimating effort including custom machine learning techniques, traditdional mathematical methods as well as providing a platform to make manual estimations and discussion between experts and managers, easier and seamless.

\newpage


\section*{Dedication}
\addcontentsline{toc}{section}{Dedication}
We dedicate this work to our parents who always supported us in everything and in every condition,
our teacher who never fails to teach and guide us, to administrators of Software Engineering
Department for their help and support throughout the degree.

\newpage


\section*{Acknowledgements}
\addcontentsline{toc}{section}{Acknowledgements}
In the name of Allah, the Most Gracious and the Most Merciful. All praises to Allah and His blessing
for the completion of this project. We thank Allah for all the opportunities, trials and strength that
have been showered on me to finish this project. We experienced so much during this process, not
only from the academic aspect but also from the aspect of personality. My humblest gratitude to the
holy Prophet Muhammad (Peace be upon him) whose way of life has been a continuous guidance for
me.


We feel highly privileged in taking this opportunity to express our heartiest gratitude to my
respective project supervisor Dr. Yasir Mehmood, Department of Software Engineering, The
University of Lahore, for his dexterous supervision, inspiring and impetuous guidance, valuable
suggestions, technical help and mostly his scolding to complete this research study as well as for
writing this dissertation.


Our deepest gratitude goes to all of our family members. It would not be possible to write this thesis
without the support from them. We also want to extend my thanks to Admin Staff in the Faculty of
Software Engineering Department, University of Lahore, for their help and support in the
administrative works.


May God shower the above cited personalities with success and honour in their life.

\newpage





% Table of contents page
\tableofcontents
\pagebreak

\listoffigures
\pagebreak



\pagenumbering{arabic}
% Chapter 1 Intorduction
\section{Introduction}

\vspace{20mm}

\Huge{\textbf{Introduction to Problem}}

\vspace{20mm}


\begin{abstract}
	In this chapter, we will be introducing the problem and the requirements that will be used to solve it. The purpose and main objectives that are at the core of the project will be explained in a concise manner. Along with those details we will also discuss the already existing solutions to the current problem and how these existing solutions are no longer a viable choice for consumers. We shall also discuss how our solution fixes the issues that were found in the existing solutions and how our system will be a superior and generally better. After these details an executive summary will summarize all of the above discussions into a concise manner.
\end{abstract}

\vspace{20mm}






\large{\textbf{Outline}}

\begin{center}
	\begin{itemize}
		\item Overview
		\item Purpose
		\item Problem Background
		\item Objectives
		\item Existing Solutions
		\item Proposed Solution
		\item Novelty
		\item Executive Summary
	\end{itemize}
\end{center}
\pagebreak








\subsection{Overview}


Software Effort Estimation (SEE) studies have started since the 1960s and continuous research has been conducted due to numerous claims on attaining accurate
estimation results. In the planning phase
of project management, SEE is an essential feature to deliver a successful software
system. Software effort estimation is defined as a process of estimating the amount of
work and hours required to develop software systems. 

Today, developing software systems are
expensive and difficult. The software engineering presents several ways to quantify a
project. One of the most important steps in software engineering process is to
accurately estimate the cost, effort and time which has an important role in determining
the success or failure of the project. The software development cost and effort
estimation are important in development process and customer requirements. The
reports on conducting projects show that there is almost no control over software
projects and usually, the scale of the accomplished work is more than what has been
estimated before. Therefore, usually projects terminate later than planned time .

According to the CHAOS report (2015) of The Standish Group International,
60\% of IT projects were not on their scheduled time and 56\% were not on budget. The
International Society of Parametric Analysis (ISPA) studied that inaccurately
estimating the staff’s skills level, underestimating software size and lack of
requirement’s understandings are some of the core reasons behind project failures. To date, researchers have therefore introduced different types of SEE
techniques. On the other hand, the majority of the techniques, were proposed at the
start of the software development process, based on pre-defined requirements.

As a solutions to all of the problems and complications above, we are introducing a solution to improve the accuracy of the process of cost and effort estimation. We are proposing a solution that will convert the manual process of guessing the project's effort, to a more standardized and reliable method of estimation. For this solution we have conducted extensive research studying many solutions and techniques to the same problem. Along with that we have also implemented our own custom machine learning models to further improve results compared to traditional formula based estimations. Using these machine learning models allows the system to make better assumptions about the project thus giving far more accurate results than traditional methods like UCP. 

The system will accept "Use Case Points" as input. Along with that the user will also have to specify the project's programming language and the used framework. This will help adjust the technical factors that influence the final calculated estimation for the project. 






\subsection{Problem Background}
Any software project’s success depends primarily on its accuracy in estimating effort. To date, a lot of research has been conducted to estimate the accuracy of software effort using distinctive techniques. In any case, researchers and specialists are striving to recognize which estimation technique gives increasingly accurate outcomes on the given datasets and the other applicable attributes. The number of software projects fails due to incomplete requirements and inaccuracy in software estimation (R. Kaur and Sengupta, 2013). The Project Management Institute (PMI) conducted a survey in 2017, investigated that 69\% of software successfully achieved the project’s original goals and business priorities, 43\% were not finished within their initial budgets, 48\% were delivered late and 32\% failed due to budget lost (PMI's.,2017).


The factors that influence effort and cost during the conception and design phases have been extensively researched, mostly using cost-estimating techniques. According to Doloi (2013), proper cost and effort estimation is the key to avoiding project cost overruns, regardless of management skill or financial strength of the contractor. Cost and effort estimation is a technical technique for predicting expenditures, and its success is dependent on the resources and project execution.
According to widely published initial estimates, project complexity, technology needs, and project team requirements are among the factors influencing cost performance.






\subsection{Purpose}
The main purpose of our system is to improve the accuracy of estimate calculations. This will be done using state-of-the-art machine learning algorithms that are trained on previous software development projects. The classifier will also improve significantly with the passage of time, since we will be recording statistics of all the projects that will be entered in the system.

This system aims to propose an ensemble model to improve the estimation accuracy of software development effort. The ensemble model is incorporated with Use Case Point (algorithmic), expert judgement (non-algorithmic) and Case-Based Reasoning (machine learning) techniques to make an ensemble.









\subsection{Objective}
Following are the most important objectives of the project.
\begin{enumerate}[(a)]
	\item To improve the accuracy of software cost and effort estimation techniques.
	\item To get managers to switch to our system by automating their current estimation workflows and then  integrating our specialized techniques with their existing workflows.
	\item To make our system compatible with most of today's trending programming languages and their popular frameworks.
	\item To enable software development companies to estimate their projects earlier in time.
\end{enumerate}













\subsection{Existing Solutions}
There are a few existing solutions in this problem space. Some of them are now not maintained and are deprecated. Following is a list of similar projects
\begin{itemize}
	\item {\bf{SEER For Software}}\newline
	SEER is a general purpose estimation tool that also has a system for software projects' estimation. It accepts input in form of SLOC (Source lines of code), function points, use cases and some more less popular options. After processing these inputs using proprietary models, the output of the program is as follows: Project time duration, development hours and accuracy of estimations.

	\item {\bf{TruePlanning \small{\textregistered}}}\newline
	This system is calculates its estimates using the PRICE model. Developed in 1975, TruePlanning is also a general purpose estimation system.
	
\end{itemize}













\subsection{Disadvantages}
Following are some serious shortcomings or disadvantages with the currently existing solutions
\begin{itemize}
	\item Outdated models that cannot adapt to today's rapidly changing and extremely diverse development languages and frameworks.
	\item No platform to enable communication between management and developers to discuss and agree to a better estimate
	\item Older and less commonly used input formats like SLOC and function points
	\item Outdated subscription methods. All of these programs implement a {\it{One Time Payment}} method. The disadvantage here is that the one time payment is a huge sum of money and may throw off a potential customer's interest in the system.
\end{itemize}









\subsection{Proposed Solution}
The proposed solution is a web-based SPA. This application will be a platform for performing all sorts of estimations for your software project. There will be two types of users, managers and developers. A manager can create a "project" and assign developers into it. The assigned developers can then give their estimates in different "rounds". After the estimates of all developers are close enough then it would regarded as the agreed estimate.

How ever if the team chooses to go towards the automatic technique, they will have to enter their project's attributes in form of "use case points". After that they will enter the "stack" that they are using to develop their project which will help set the environmental factors and technical factors. The system will then calculate the estimates for the project using either the simple UCP technique or the machine learning methods (according to the user's preferences). The estimates will be displayed in a various forms like tables, graphs and charts.


\subsection{Comparison}
We have compiled a comprehensive list of important issues in the existing solutions. After stating the problem we have also proved how our proposed system solves that problem. Following is a list of problem and their solutions

\begin{center}

	\large{\textbf{Existing Solutions}} \hspace{30mm} \large{\textbf{Proposed Solution}}
	\vspace{5mm}
	\large

	\begin{tabularx}{\textwidth}{|X|X|}

		\hline
		{\textbf{1)} Use older estimations models are used that can no longer accurately estimate today's versatile development technologies} & {\textbf{1)} Combines a variety of well proven estimations models into an ensemble model that is trained on historical data} \\
		\hline
		{\textbf{2)} No multi model support, all calculations are based on one single algorithm hence single point of failure is applicable} & {\textbf{2)} Has multiple methods of estimation including manual methods as well as algorithmic and machine learning methods} \\
		\hline
		{\textbf{3)} Less accurate estimations. Since these programs have not been update according to modern development practices, their accuracies have degraded overtime} & {\textbf{3)} Uses state of the art estimation models that are trained on modern datasets, hence accuracy is far better} \\
		\hline
		{\textbf{4)} Don't come with many features. No platform to implement manual estimation techniques of effort estimation } & {\textbf{4)} Provides platform to enable communication between management and developers to discuss and agree to a better estimate} \\
		\hline
		

	\end{tabularx}
\end{center}





\subsection{Novelty}
Here are some points that make our system a better solution than the existing ones mentioned above.

\begin{itemize}
	\item {\bfseries Latest data:} Our calculations and estimates are based on the data that is collected from the past few years. This data is obtained from the UCP dataset. Along with that, we have also expanded the UCP dataset from 72 to 99 projects by collecting data from some development organizations. Newer projects are created using latest technologies and frameworks hence they are a better fit to compare with the future projects. Also the system itself will be collecting data from all of it's customer for the sake of self improvement.
	
	
	\item {\bfseries Machine Learning:} In the past all manual or automated estimations were done using simple mathematical formulas and techniques. This approach is very rigid as it doesn't leave much room for flexibility. With machine learning techniques, no two estimations are the same. The system will be able to learn from the past data and make better estimates.
	

	\item {\bfseries Platform to collaborate:} All of the cost estimation services will be provided on a platform that is built for the collaboration of developers and managers. Here you can estimate your project and along with that, get statistics of your project. You can manage all your project at one place. Generate graphs and tables of your project's estimates to extract helpful insights from it.
	

	\item {\bfseries Multiple Methods:} Initially we will be providing three different techniques for estimations. First is the "Delphi Technique" for manual estimations, this is where all the assigned developers will partake in "rounds" to agree on an estimate. Second is the "UCP Technique" for estimating the project using the use case points. This is traditional approach but will be provided to give the users more options. Third is the "Machine Learning" technique. This is the most advanced technique and has been tested to provide the best results in {\it{most}} cases. This freedom of choice will improve user experience by letting them choose the best approach for themselves.
\end{itemize}


\subsection{Significance}
In this section we will be discussing the major impacts of our systems on the software development process as well as how it will affect software development industries.





\subsection{Summary}
The proposed system aspires to revolutionize and update the methodologies of software cost and effort estimation. This particular area of software engineering hasn't been worked on in order to create an effective and usable system. The Proposed system implements state of the art machine learning algorithms to give personalized and highly accurate estimations to all sorts of software projects. This accuracy is achieved by training the model on modern datasets collected from actual software development companies of successfully completed projects.This system will provide multiple methods to calculate estimates such as the Delphi Technique, UCP Technique and Machine Learning techniques.


Along with the automated cost estimation services, the system will also provide a platform to collaborate with developers and managers to discuss and agree on a better estimate. This platform is implemented to enable the use of manual estimation techniques such as Delphi.






% Chapter 2 Software Requirements Specification

\section{Software Requirements Specification}

\vspace{20mm}



\begin{abstract}
	In this chapter,we will discuss a software development process which is generally consists of several stages ranging from analysis
    to implementation, which will result in need (requirement) of software. Results of analysis for different
    applications in a similar problem domain will usually produce similar definition needs. In the field of 
    software engineering, especially in object- oriented software development, requirements for different problem
    domain have been developed. This research is to obtain a similarity (commonality) of the software requirements
    specifications, here in after in short-SRS. Management Information System (MIS) has been chosen due to several
    years has developed into concepts that are essential both in the scope of major agencies as well as small and medium
    scale. SRS is the identification of commonality for the MIS application which is focused on the function of customer
    service standards, particularly for the Management Information System. From this research it also can be concluded
    that the SRS commonality can be generated by capturing best practices from existing business processes.
    
\end{abstract}

\vspace{20mm}

\large{\textbf{Outline}}

\begin{center}
    \begin{itemize}
        \item Introduction
              \begin{itemize}
                  \item Purpose
                  \item Scope
              \end{itemize}
        \item Overall Description
        \item User Characteristics
        \item Specific Requirements
              \begin{itemize}
                  \item Functional Requirements
                  \item Non-Functional Requirements
              \end{itemize}
    \end{itemize}
\end{center}
\pagebreak


\subsection{Introduction}

A software requirements specification SRS describes all the requirements system which must-have for success.
These requirements are typically illustrating features of an underdevelopment system. These features not only
describe its functional requirements FR but also its nonfunctional requirements NFR. Functional requirements
as well nonfunctional requirements are equally significant for successful development. Both FR and NFR have
an impact on the overall success of any system. In contrast to FRs which are described deliberately in SRS,
NFRs are not described explicitly. If some NFRs are described deliberately, there are multiple NFRs implicitly
defined hidden in plain text. Encoding software requirements from the earliest period of development are 
more focused on functional requirements.

\hspace{10mm}Non-functional requirements either neglected or described as a whole
i.e., not for the individual design problem. These requirements artifacts are encoded without any prescribed
standards, it can be either structured, semi-structured, or completely unstructured. To learn
these non-functional requirements, we have to understand the genre of requirements artifacts are organized 
such as Modules, submodules, and finally atomic design problems. These design problems are not defined in any
structured way most of the time requirement encoders spent more time on functional requirements. These 
requirements in a textual form commonly describe functional requirements, however, implicitly define non-functional
requirements as well. The textual data for requirement extraction can be used for the extraction of underlying functional
requirements. To learn these underlying functional requirements a standardized way of encoding, functional
requirements must be developed. Requirements artifacts are encoded using the wording of end-users i.e. interviews 
or the words of requirement from requirement inceptor i.e. owner, end-users.


 \subsection{Purpose}
 Software requirement artifacts such as manuals request for proposals, and software requirements specification (SRS) are
 commonly focused on functional requirements. In most SRS files, nonfunctional requirements do not formally encoded or 
 encoded as a whole, not for an individual design problem. Moreover, these nonfunctional requirements are intermingled 
 with functional requirements. Therefore, these nonfunctional requirements need special attention to understand for
 successful project development. These nonfunctional requirements have an impact on each other and optimal tradeoff is
 required for balanced nonfunctional requirements set. NFRs have a negative and positive tradeoff with each other such
 as increase confidentiality, decrease the availability, and enhance authenticity.So, an optimum tradeoff among these 
 design problem within a module is required to have better design decisions. 
 
 \hspace{10mm}Instead of considering all nonfunctional
 requirements, the NFRs that have mutual tradeoff is considered. In this paper, we devised a novel document annotation
 scheme for SRS and extracted nonfunctional requirements from these annotated artifacts. In the next step, we classified
 NFRs into two classes security triad and performance triad, and the cost is assumed constant for each NFR. From the design
 problem, the tradeoff ratio is calculated among NFRs associated with it. Then, the production possibility graph is plotted
 to estimate the optimum tradeoff ratio within the module. For estimation economic optimum from a set of NFR, iso-cost graphs
 by assuming the constant cost. Some hypothetical variations in cost are also examined using 3D iso-cost graph. The reason to
 measure these tradeoff is to make design decision more empirical and helpful for the selection of design patterns, especially
 secure design patterns.

    \subsection{Product Scope}
    The major Scope of our Project is to estimate the actual cost of the project and helpful to managers of the software industry
    are likely easy to calculate the budget of the developing software, and also easy to assign the developer to the particular 
    developing software. This application helps Software Associations for dealing with an upcoming project in terms of their cost 
    as well as effort applied for that project. So they easily decide that they have to work on that project or not as well as they also
    know that either they have the developer of that technology or not so they recruited that type of programmers.
    \\
    
    \subsubsection{Definition acronym \& abbrevation}
    The following acronym are used in our documents
    \begin{center}
        \begin{itemize}
            \item {\bfseries MERN:} MongoDB ExpressJS ReactJS NodeJS
            \item {\bfseries REST:} Restfull
            \item {\bfseries API:} Application Programming Interface
            \item {\bfseries DB:} Database
            \item {\bfseries SRS:} Software Requirements Specification
            \item {\bfseries UCP:} Use Case Point
            \item {\bfseries SPA:} Single Page Application
            \item {\bfseries CBR:} Content Base Reasoning
            \item {\bfseries FRs:} Functional Requirements
            \item {\bfseries NFRs:} Non-Functional Requirements
            \end{itemize}
    \end{center}
    
\subsection{Overall Description}
\subsubsection{Product Perspective}
Our Final Year Project is SPA(Single Page Application). To reduce the difficulties for those managers who feels 
difficulties to estimate the actual effort/cost of the upcoming project. Managers can also some other modules except
the Estimation i.e {\bfseries View/Update project} and {\bfseries Add/Remove Developers} in a project. Other than manager, there is a developer module in which developer 
can View and inform estimation of the project. Manager also have option to use our all methods for estimation as well as choose only one or two for calculating the actual estimate of the project.

    \subsubsection{System Interfaces:}
    Our Proposed system is built from the scratch from the modern technology and add one feature from the exciting application. 
    \subsubsection{User interface: }
    The interface of the website is designed to be simple and interactive and be Single Page Application to improve the user experience. 
    For attracting the developer and managers, we have facilitate them to change the theme of the system as they can feel to easy for work.
    As we know who attract with our system has well knowledge of how to operate a website but we kept as easy to operatable.
    \subsubsection{Prototype}
    {\huge Prototype lgani ha yahan}
    \subsubsection{Hardware Interfaces}
    The website can be used on any device such as a mobile phone laptop, computer or a tab as long as it has active Internet connection.
    \subsubsection{Software Interface}
    Software interfaces include operating system for the computer, a browser to access the website with internet connectivity.
\subsection{Functional Requiremets}
    \newcommand{\splitcell}[2][c]{%
  \begin{tabular}[#1]{@{}c@{}}#2\end{tabular}}



\paragraph{FR-01}
\begin{center}
    \begin{tabular}{ |c|c|c|c|c|}
        \hline
        \textbf{ID} & \multicolumn{4}{c|}{\textbf{FR-01}} \\
        \hline
        Name & \multicolumn{4}{c|}{Register} \\
        \hline
        \textbf{Description} & \textbf{Input} & \textbf{Output} & \textbf{Requirements} & \textbf{Basic Workflow} \\
        \hline
        \splitcell{Enter details and \\ register} & \splitcell{Name, Email, Password,\\ Phone Number} & \splitcell{Succesfully \\ registered} & \splitcell{Internet \\ connectivity} & \splitcell{
            Enter details that \\ are required and click \\ on the button to register.
        } \\
        \hline
    \end{tabular}
\end{center}







\paragraph{FR-02}
\begin{center}
    \begin{tabular}{ |c|c|c|c|c|}
        \hline
        \textbf{ID} & \multicolumn{4}{c|}{\textbf{FR-01}} \\
        \hline
        Name & \multicolumn{4}{c|}{Register} \\
        \hline
        \textbf{Description} & \textbf{Input} & \textbf{Output} & \textbf{Requirements} & \textbf{Basic Workflow} \\
        \hline
        \splitcell{Enter details and \\ register} & \splitcell{Name, Email, Password,\\ Phone Number} & \splitcell{Succesfully \\ registered} & \splitcell{Internet \\ connectivity} & \splitcell{
            Enter details that \\ are required and click \\ on the button to register.
        } \\
        \hline
    \end{tabular}
\end{center}


    


\subsection{User Characteristics}
\subsection*{Application User}
After the compeletion of project, application user will able to estimate the budget as well as effort of the new project and later on it will be save for the future if any similar project will be arrive they make sure the effort of that type of software will be this.
Application User will easily entertain their client with the almost accurate budget. 
\subsubsection{Constraint}
The application need an internet access it will not work without internet access.
\subsubsection{Assumptions and dependencies}
\begin{center}
    \begin{itemize}
        \item This application will target the Managers of the Software Associations to find the almost actual effort as well as budget of the upcoming project.
        \item User should know the basic knowledge of the system.
    \end{itemize}
\end{center}
\subsection{Non Functional Requirements}
\subsubsection{Usability}
Usability of the system is high because it is easy to use. Names of choices are very clear and according to their functionality. Its interface is not difficult as user can understand
everything clearly and use its functionality without any confusion.  
\subsubsection{Maintainability}
System will be developed module wise so Maintainability will be achieved by architecture.
\subsubsection{Reusability}
The application designed so that its code can be reused in other application similar to this application and as well as in other applications.
\paragraph{Response Time}
Response time depends on the speed of internet of user. Higher the spped of internet higher will be the reponse time.





% Chapter 3 Use Case Analysis
% {
  name: 'Register',
  desc: '',
  actors: '',
  assumptions: '',
  triggers: '',
  pre: '',
  post: '',
  mainCourse: '',
  alternativeCourse: ''
}

    \subsubsection{FR-01: Register}
    \begin{center}
        \begin{tabularx}{\textwidth}{|l|X|}
            \hline
            \textbf{ID} & UC-01 \\
            \hline
            \textbf{Name} & Register \\
            \hline
            \textbf{Description} &  \\
            \hline
            \textbf{Actors} &  \\
            \hline
            \textbf{Assumptions} &  \\
            \hline
            \textbf{Triggers} &  \\
            \hline
            \textbf{Pre Conditions} &  \\
            \hline
            \textbf{Post Conditions} &  \\
            \hline
            \textbf{Main Course} &  \\
            \hline
            \textbf{Alternative Course} &  \\
            \hline
            
        \end{tabularx}
    \end{center}
    
    

    \subsubsection{FR-02: Register}
    \begin{center}
        \begin{tabularx}{\textwidth}{|l|X|}
            \hline
            \textbf{ID} & UC-02 \\
            \hline
            \textbf{Name} & Register \\
            \hline
            \textbf{Description} & User can register their to keep their role in that software \\
            \hline
            \textbf{Actors} & New User \\
            \hline
            \textbf{Assumptions} & If user can Register, he/she will see more features \\
            \hline
            \textbf{Triggers} & Just Connect to in the internet and a email address \\
            \hline
            \textbf{Pre Conditions} & None \\
            \hline
            \textbf{Post Conditions} & Account Created Succesfully \\
            \hline
            \textbf{Main Course} & 1. Enter their Details 2.Create their accound according to their given details \\
            \hline
            \textbf{Alternative Course} & Error due to invalid details \\
            \hline
            
        \end{tabularx}
    \end{center}
    
    

    \subsubsection{FR-03: Login}
    \begin{center}
        \begin{tabularx}{\textwidth}{|l|X|}
            \hline
            \textbf{ID} & UC-03 \\
            \hline
            \textbf{Name} & Login \\
            \hline
            \textbf{Description} & Login to the system to view/change in their account \\
            \hline
            \textbf{Actors} & New User \\
            \hline
            \textbf{Assumptions} & If user can login, he/she will use more features \\
            \hline
            \textbf{Triggers} & just confirm with email which they provided in sign up \\
            \hline
            \textbf{Pre Conditions} & they must have a account \\
            \hline
            \textbf{Post Conditions} & they succesfully see the dashboard and more settings and features \\
            \hline
            \textbf{Main Course} & User can enter/use their email and its valid password Check entered data and able to use \\
            \hline
            \textbf{Alternative Course} & Error due to invalid details \\
            \hline
            
        \end{tabularx}
    \end{center}
    
    

    \subsubsection{FR-04: OAuth}
    \begin{center}
        \begin{tabularx}{\textwidth}{|l|X|}
            \hline
            \textbf{ID} & UC-04 \\
            \hline
            \textbf{Name} & OAuth \\
            \hline
            \textbf{Description} & User can ouath their details with third party like Gmail , LinkdIn and Github \\
            \hline
            \textbf{Actors} & New User \\
            \hline
            \textbf{Assumptions} & User can connect their third party account to login with the software \\
            \hline
            \textbf{Triggers} & just have a third party account  \\
            \hline
            \textbf{Pre Conditions} & none \\
            \hline
            \textbf{Post Conditions} & account has created by confiramtion through email  \\
            \hline
            \textbf{Main Course} & User can create their account on system by sending a email confirmation token. \\
            \hline
            \textbf{Alternative Course} & Error will show if you cant verify your Thir party account \\
            \hline
            
        \end{tabularx}
    \end{center}
    \newpage
    

    \subsubsection{FR-05: Forget Password}
    \begin{center}
        \begin{tabularx}{\textwidth}{|l|X|}
            \hline
            \textbf{ID} & UC-05 \\
            \hline
            \textbf{Name} & Forget Password \\
            \hline
            \textbf{Description} & Send the email to the user with a link to reset their password \\
            \hline
            \textbf{Actors} & Account Holder \\
            \hline
            \textbf{Assumptions} & An email is sent tu the user with a link to reset their password \\
            \hline
            \textbf{Triggers} &  \\
            \hline
            \textbf{Pre Conditions} & Valid Email \\
            \hline
            \textbf{Post Conditions} & New password has been changed \\
            \hline
            \textbf{Main Course} & User can change their password by sending link to their email and setting the new password with the password requirements \\
            \hline
            \textbf{Alternative Course} & Error will be show due to invalid details \\
            \hline
            
        \end{tabularx}
    \end{center}
    
    

    \subsubsection{FR-06: Reset Password}
    \begin{center}
        \begin{tabularx}{\textwidth}{|l|X|}
            \hline
            \textbf{ID} & UC-06 \\
            \hline
            \textbf{Name} & Reset Password \\
            \hline
            \textbf{Description} & User can reset your Password by entering you current password \\
            \hline
            \textbf{Actors} & Account Holder \\
            \hline
            \textbf{Assumptions} & To reset their password, User can enter the curent password and after confirmation it will be default password  \\
            \hline
            \textbf{Triggers} & Account Holder \\
            \hline
            \textbf{Pre Conditions} & user have a account \\
            \hline
            \textbf{Post Conditions} & password has been reset \\
            \hline
            \textbf{Main Course} & User can reset their password by entering their current password and password will set to default Password \\
            \hline
            \textbf{Alternative Course} & Error will be show due to invalid details \\
            \hline
            
        \end{tabularx}
    \end{center}
    
    

    \subsubsection{FR-07: Confirm Account}
    \begin{center}
        \begin{tabularx}{\textwidth}{|l|X|}
            \hline
            \textbf{ID} & UC-07 \\
            \hline
            \textbf{Name} & Confirm Account \\
            \hline
            \textbf{Description} & User can confirm their account by a valid email address \\
            \hline
            \textbf{Actors} & New User \\
            \hline
            \textbf{Assumptions} &  \\
            \hline
            \textbf{Triggers} & By send the valid link to user's provided email \\
            \hline
            \textbf{Pre Conditions} & Fill the signup form \\
            \hline
            \textbf{Post Conditions} & account verified \\
            \hline
            \textbf{Main Course} & Account has been created though 3rd Party validation \\
            \hline
            \textbf{Alternative Course} & Error by unvalid email provided \\
            \hline
            
        \end{tabularx}
    \end{center}
    
    

    \subsubsection{FR-08: Delete Account}
    \begin{center}
        \begin{tabularx}{\textwidth}{|l|X|}
            \hline
            \textbf{ID} & UC-08 \\
            \hline
            \textbf{Name} & Delete Account \\
            \hline
            \textbf{Description} & User can delete the account if he/she can \\
            \hline
            \textbf{Actors} & Already User \\
            \hline
            \textbf{Assumptions} &  \\
            \hline
            \textbf{Triggers} & remove the user\_id from database by click on reset button \\
            \hline
            \textbf{Pre Conditions} & Already user database was be present in database \\
            \hline
            \textbf{Post Conditions} & No user of that user\_id will not be in database \\
            \hline
            \textbf{Main Course} & Account has been removed from database by the confirmation of user \\
            \hline
            \textbf{Alternative Course} & Error will be appeared by database \\
            \hline
            
        \end{tabularx}
    \end{center}
    \newpage
    

    \subsubsection{FR-09: Edit Account Info}
    \begin{center}
        \begin{tabularx}{\textwidth}{|l|X|}
            \hline
            \textbf{ID} & UC-09 \\
            \hline
            \textbf{Name} & Edit Account Info \\
            \hline
            \textbf{Description} & User can edit their account information \\
            \hline
            \textbf{Actors} & Already User \\
            \hline
            \textbf{Assumptions} &  \\
            \hline
            \textbf{Triggers} & Database of that user can be updated \\
            \hline
            \textbf{Pre Conditions} & Already data of that user will be in database  \\
            \hline
            \textbf{Post Conditions} & Update the new version of their information in database \\
            \hline
            \textbf{Main Course} & Information of the User can be updated if he/she can \\
            \hline
            \textbf{Alternative Course} & Error willbe displaced by the Database \\
            \hline
            
        \end{tabularx}
    \end{center}
    
    

    \subsubsection{FR-10: Edit Interface Preferences}
    \begin{center}
        \begin{tabularx}{\textwidth}{|l|X|}
            \hline
            \textbf{ID} & UC-10 \\
            \hline
            \textbf{Name} & Edit Interface Preferences \\
            \hline
            \textbf{Description} & User can edit their preferences of the UI of their dashboard \\
            \hline
            \textbf{Actors} & User \\
            \hline
            \textbf{Assumptions} &  \\
            \hline
            \textbf{Triggers} & User can change the primary and secondary Colors of the themes \\
            \hline
            \textbf{Pre Conditions} & Valid User \\
            \hline
            \textbf{Post Conditions} & UI preferences will be change according to User \\
            \hline
            \textbf{Main Course} & Interface preference will be changed by the user according to their preception  \\
            \hline
            \textbf{Alternative Course} & Error will be displayed it choose certain colors like black etc. \\
            \hline
            
        \end{tabularx}
    \end{center}
    
    

    \subsubsection{FR-11: Create Project}
    \begin{center}
        \begin{tabularx}{\textwidth}{|l|X|}
            \hline
            \textbf{ID} & UC-11 \\
            \hline
            \textbf{Name} & Create Project \\
            \hline
            \textbf{Description} & Manager can create a project which is stored in the database with the unique id \\
            \hline
            \textbf{Actors} & Manager \\
            \hline
            \textbf{Assumptions} &  \\
            \hline
            \textbf{Triggers} & The Project will created in the database with unique id \\
            \hline
            \textbf{Pre Conditions} & user will be valid and Project can't created before  \\
            \hline
            \textbf{Post Conditions} & Project is created by the Manager \\
            \hline
            \textbf{Main Course} & Project is created by the manager in database \\
            \hline
            \textbf{Alternative Course} & Error will be displayed of database  \\
            \hline
            
        \end{tabularx}
    \end{center}
    
    

    \subsubsection{FR-12: Edit Project}
    \begin{center}
        \begin{tabularx}{\textwidth}{|l|X|}
            \hline
            \textbf{ID} & UC-12 \\
            \hline
            \textbf{Name} & Edit Project \\
            \hline
            \textbf{Description} & The existing project will be edited by the manager if needed  \\
            \hline
            \textbf{Actors} & Manager \\
            \hline
            \textbf{Assumptions} &  \\
            \hline
            \textbf{Triggers} & The Information of the existing project will be updated in the database \\
            \hline
            \textbf{Pre Conditions} & Project is present already in database \\
            \hline
            \textbf{Post Conditions} & Project's Information will be updated by manager \\
            \hline
            \textbf{Main Course} & Manager can change the existing project if the requirements will be changed by the client \\
            \hline
            \textbf{Alternative Course} & Database's error will occurs \\
            \hline
            
        \end{tabularx}
    \end{center}
    \newpage
    

    \subsubsection{FR-13: Delete Project}
    \begin{center}
        \begin{tabularx}{\textwidth}{|l|X|}
            \hline
            \textbf{ID} & UC-13 \\
            \hline
            \textbf{Name} & Delete Project \\
            \hline
            \textbf{Description} & The Manager will delete the existing projects from the database \\
            \hline
            \textbf{Actors} & Managers \\
            \hline
            \textbf{Assumptions} &  \\
            \hline
            \textbf{Triggers} & Delete the existing project from the database by delete event \\
            \hline
            \textbf{Pre Conditions} & Project will be in the database which Manager will be deleted \\
            \hline
            \textbf{Post Conditions} & Project is deletedby the the manager \\
            \hline
            \textbf{Main Course} & The manager will delete the exiting project which is presnet in database \\
            \hline
            \textbf{Alternative Course} & Error will be displayed \\
            \hline
            
        \end{tabularx}
    \end{center}
    
    

    \subsubsection{FR-14: Add Developer}
    \begin{center}
        \begin{tabularx}{\textwidth}{|l|X|}
            \hline
            \textbf{ID} & UC-14 \\
            \hline
            \textbf{Name} & Add Developer \\
            \hline
            \textbf{Description} & The Manager will assign the developer of the company which is part of that Project \\
            \hline
            \textbf{Actors} & Managers \\
            \hline
            \textbf{Assumptions} &  \\
            \hline
            \textbf{Triggers} & Developer list will be shown to manager and manager will add the relevant developer \\
            \hline
            \textbf{Pre Conditions} & Developer and project will be added \\
            \hline
            \textbf{Post Conditions} & Developer will be added in the specific project \\
            \hline
            \textbf{Main Course} & To view the new upcoming project and take the decision of developer in project, Manager will added the developer in the project \\
            \hline
            \textbf{Alternative Course} & Error will be displayed \\
            \hline
            
        \end{tabularx}
    \end{center}
    
    

    \subsubsection{FR-15: Remove Developer}
    \begin{center}
        \begin{tabularx}{\textwidth}{|l|X|}
            \hline
            \textbf{ID} & UC-15 \\
            \hline
            \textbf{Name} & Remove Developer \\
            \hline
            \textbf{Description} & The manager can remove the developer after developer performs their duties \\
            \hline
            \textbf{Actors} & Manager \\
            \hline
            \textbf{Assumptions} &  \\
            \hline
            \textbf{Triggers} & Manager will remove the developer in the project anytime. \\
            \hline
            \textbf{Pre Conditions} & Developer will be added in that project \\
            \hline
            \textbf{Post Conditions} & developer is no more the part of the project \\
            \hline
            \textbf{Main Course} & Manager can manage the availabilty of the developer whenever he enter in the project or when he exits. \\
            \hline
            \textbf{Alternative Course} & Error will be displayed \\
            \hline
            
        \end{tabularx}
    \end{center}
    
    

    \subsubsection{FR-16: Edit Developer Permission}
    \begin{center}
        \begin{tabularx}{\textwidth}{|l|X|}
            \hline
            \textbf{ID} & UC-16 \\
            \hline
            \textbf{Name} & Edit Developer Permission \\
            \hline
            \textbf{Description} & The Manager can edit the role of the developer in the project \\
            \hline
            \textbf{Actors} & Managers \\
            \hline
            \textbf{Assumptions} &  \\
            \hline
            \textbf{Triggers} & Manager can edit the preference of the developer \\
            \hline
            \textbf{Pre Conditions} & Developer must have the part of the project \\
            \hline
            \textbf{Post Conditions} & Developer's Role will changed by the manager \\
            \hline
            \textbf{Main Course} & The Developer can change the role of the developer as he needs him in the project \\
            \hline
            \textbf{Alternative Course} & Error will be dislpayed by the database \\
            \hline
            
        \end{tabularx}
    \end{center}
    \newpage
    

    \subsubsection{FR-17: Calculate UCP}
    \begin{center}
        \begin{tabularx}{\textwidth}{|l|X|}
            \hline
            \textbf{ID} & UC-17 \\
            \hline
            \textbf{Name} & Calculate UCP \\
            \hline
            \textbf{Description} & Calculate the effort estimation of project via mean of UCP method \\
            \hline
            \textbf{Actors} & Manager , Developers \\
            \hline
            \textbf{Assumptions} &  \\
            \hline
            \textbf{Triggers} & The effort will calculted by UCP formula \\
            \hline
            \textbf{Pre Conditions} & The Project Use cases' information are inserted already \\
            \hline
            \textbf{Post Conditions} & Calculated Effort are given by UCP method  \\
            \hline
            \textbf{Main Course} & To find the Effort Estimation by UCP method of the entire project by the help of their use cases \\
            \hline
            \textbf{Alternative Course} & Error will displayed \\
            \hline
            
        \end{tabularx}
    \end{center}
    
    

    \subsubsection{FR-18: Get Machine Learning Estimation}
    \begin{center}
        \begin{tabularx}{\textwidth}{|l|X|}
            \hline
            \textbf{ID} & UC-18 \\
            \hline
            \textbf{Name} & Get Machine Learning Estimation \\
            \hline
            \textbf{Description} & Calculate the effort estimation of project via trained Machine Learning module \\
            \hline
            \textbf{Actors} & Manager  \\
            \hline
            \textbf{Assumptions} &  \\
            \hline
            \textbf{Triggers} & The effort will calculted by UCP formula \\
            \hline
            \textbf{Pre Conditions} & The Project Use cases' information are inserted already \\
            \hline
            \textbf{Post Conditions} & Calculated Effort are given by UCP method  \\
            \hline
            \textbf{Main Course} & To find the Effort Estimation by UCP method of the entire project by the help of their use cases \\
            \hline
            \textbf{Alternative Course} & Error will displayed \\
            \hline
            
        \end{tabularx}
    \end{center}
    
    

    \subsubsection{FR-19: Manual Estimate Round}
    \begin{center}
        \begin{tabularx}{\textwidth}{|l|X|}
            \hline
            \textbf{ID} & UC-19 \\
            \hline
            \textbf{Name} & Manual Estimate Round \\
            \hline
            \textbf{Description} & Calculate the effort estimation of project via Manual Estimation where the developer and experts will estimate the project while round \\
            \hline
            \textbf{Actors} & developers, manager \\
            \hline
            \textbf{Assumptions} &  \\
            \hline
            \textbf{Triggers} & The Effort will calculated by Manual technique \\
            \hline
            \textbf{Pre Conditions} & Developers were added by manager  in the project \\
            \hline
            \textbf{Post Conditions} & The effort will calculated after the many rounds of the experts' discussions  \\
            \hline
            \textbf{Main Course} & To find the Effort estimation by Manual technique by many rounds \\
            \hline
            \textbf{Alternative Course} & Error will displayed \\
            \hline
            
        \end{tabularx}
    \end{center}
    
    

    \subsubsection{FR-20: Estimation Bar}
    \begin{center}
        \begin{tabularx}{\textwidth}{|l|X|}
            \hline
            \textbf{ID} & UC-20 \\
            \hline
            \textbf{Name} & Estimation Bar \\
            \hline
            \textbf{Description} & All Estimation will be shown in the form of graphs \\
            \hline
            \textbf{Actors} & Manager , Developer \\
            \hline
            \textbf{Assumptions} &  \\
            \hline
            \textbf{Triggers} & The Estimate will be generated by the experts' round discussion \\
            \hline
            \textbf{Pre Conditions} & The estimation will be measured before that step  \\
            \hline
            \textbf{Post Conditions} & The charts will be appeared \\
            \hline
            \textbf{Main Course} & Estimation will be deliver in bar that anyone will be see and clearly judge the estimation of the project \\
            \hline
            \textbf{Alternative Course} & Error wil be displayed \\
            \hline
            
        \end{tabularx}
    \end{center}
    \newpage
    


\section{Use Case Analysis}

\vspace{20mm}

\begin{abstract}
	\blindtext[3]
    
\end{abstract}


\vspace{20mm}

\large{\textbf{Outline}}

\begin{center}
    \begin{itemize}
        \item Manager Use Cases
        \item Developer Use Cases
        \item Miscellaneous Use Cases
    \end{itemize}
\end{center}

\pagebreak

\subsection{Use Case Analysis}
{
  name: 'Register',
  desc: '',
  actors: '',
  assumptions: '',
  triggers: '',
  pre: '',
  post: '',
  mainCourse: '',
  alternativeCourse: ''
}

    \subsubsection{FR-01: Register}
    \begin{center}
        \begin{tabularx}{\textwidth}{|l|X|}
            \hline
            \textbf{ID} & UC-01 \\
            \hline
            \textbf{Name} & Register \\
            \hline
            \textbf{Description} &  \\
            \hline
            \textbf{Actors} &  \\
            \hline
            \textbf{Assumptions} &  \\
            \hline
            \textbf{Triggers} &  \\
            \hline
            \textbf{Pre Conditions} &  \\
            \hline
            \textbf{Post Conditions} &  \\
            \hline
            \textbf{Main Course} &  \\
            \hline
            \textbf{Alternative Course} &  \\
            \hline
            
        \end{tabularx}
    \end{center}
    
    

    \subsubsection{FR-02: Register}
    \begin{center}
        \begin{tabularx}{\textwidth}{|l|X|}
            \hline
            \textbf{ID} & UC-02 \\
            \hline
            \textbf{Name} & Register \\
            \hline
            \textbf{Description} & User can register their to keep their role in that software \\
            \hline
            \textbf{Actors} & New User \\
            \hline
            \textbf{Assumptions} & If user can Register, he/she will see more features \\
            \hline
            \textbf{Triggers} & Just Connect to in the internet and a email address \\
            \hline
            \textbf{Pre Conditions} & None \\
            \hline
            \textbf{Post Conditions} & Account Created Succesfully \\
            \hline
            \textbf{Main Course} & 1. Enter their Details 2.Create their accound according to their given details \\
            \hline
            \textbf{Alternative Course} & Error due to invalid details \\
            \hline
            
        \end{tabularx}
    \end{center}
    
    

    \subsubsection{FR-03: Login}
    \begin{center}
        \begin{tabularx}{\textwidth}{|l|X|}
            \hline
            \textbf{ID} & UC-03 \\
            \hline
            \textbf{Name} & Login \\
            \hline
            \textbf{Description} & Login to the system to view/change in their account \\
            \hline
            \textbf{Actors} & New User \\
            \hline
            \textbf{Assumptions} & If user can login, he/she will use more features \\
            \hline
            \textbf{Triggers} & just confirm with email which they provided in sign up \\
            \hline
            \textbf{Pre Conditions} & they must have a account \\
            \hline
            \textbf{Post Conditions} & they succesfully see the dashboard and more settings and features \\
            \hline
            \textbf{Main Course} & User can enter/use their email and its valid password Check entered data and able to use \\
            \hline
            \textbf{Alternative Course} & Error due to invalid details \\
            \hline
            
        \end{tabularx}
    \end{center}
    
    

    \subsubsection{FR-04: OAuth}
    \begin{center}
        \begin{tabularx}{\textwidth}{|l|X|}
            \hline
            \textbf{ID} & UC-04 \\
            \hline
            \textbf{Name} & OAuth \\
            \hline
            \textbf{Description} & User can ouath their details with third party like Gmail , LinkdIn and Github \\
            \hline
            \textbf{Actors} & New User \\
            \hline
            \textbf{Assumptions} & User can connect their third party account to login with the software \\
            \hline
            \textbf{Triggers} & just have a third party account  \\
            \hline
            \textbf{Pre Conditions} & none \\
            \hline
            \textbf{Post Conditions} & account has created by confiramtion through email  \\
            \hline
            \textbf{Main Course} & User can create their account on system by sending a email confirmation token. \\
            \hline
            \textbf{Alternative Course} & Error will show if you cant verify your Thir party account \\
            \hline
            
        \end{tabularx}
    \end{center}
    \newpage
    

    \subsubsection{FR-05: Forget Password}
    \begin{center}
        \begin{tabularx}{\textwidth}{|l|X|}
            \hline
            \textbf{ID} & UC-05 \\
            \hline
            \textbf{Name} & Forget Password \\
            \hline
            \textbf{Description} & Send the email to the user with a link to reset their password \\
            \hline
            \textbf{Actors} & Account Holder \\
            \hline
            \textbf{Assumptions} & An email is sent tu the user with a link to reset their password \\
            \hline
            \textbf{Triggers} &  \\
            \hline
            \textbf{Pre Conditions} & Valid Email \\
            \hline
            \textbf{Post Conditions} & New password has been changed \\
            \hline
            \textbf{Main Course} & User can change their password by sending link to their email and setting the new password with the password requirements \\
            \hline
            \textbf{Alternative Course} & Error will be show due to invalid details \\
            \hline
            
        \end{tabularx}
    \end{center}
    
    

    \subsubsection{FR-06: Reset Password}
    \begin{center}
        \begin{tabularx}{\textwidth}{|l|X|}
            \hline
            \textbf{ID} & UC-06 \\
            \hline
            \textbf{Name} & Reset Password \\
            \hline
            \textbf{Description} & User can reset your Password by entering you current password \\
            \hline
            \textbf{Actors} & Account Holder \\
            \hline
            \textbf{Assumptions} & To reset their password, User can enter the curent password and after confirmation it will be default password  \\
            \hline
            \textbf{Triggers} & Account Holder \\
            \hline
            \textbf{Pre Conditions} & user have a account \\
            \hline
            \textbf{Post Conditions} & password has been reset \\
            \hline
            \textbf{Main Course} & User can reset their password by entering their current password and password will set to default Password \\
            \hline
            \textbf{Alternative Course} & Error will be show due to invalid details \\
            \hline
            
        \end{tabularx}
    \end{center}
    
    

    \subsubsection{FR-07: Confirm Account}
    \begin{center}
        \begin{tabularx}{\textwidth}{|l|X|}
            \hline
            \textbf{ID} & UC-07 \\
            \hline
            \textbf{Name} & Confirm Account \\
            \hline
            \textbf{Description} & User can confirm their account by a valid email address \\
            \hline
            \textbf{Actors} & New User \\
            \hline
            \textbf{Assumptions} &  \\
            \hline
            \textbf{Triggers} & By send the valid link to user's provided email \\
            \hline
            \textbf{Pre Conditions} & Fill the signup form \\
            \hline
            \textbf{Post Conditions} & account verified \\
            \hline
            \textbf{Main Course} & Account has been created though 3rd Party validation \\
            \hline
            \textbf{Alternative Course} & Error by unvalid email provided \\
            \hline
            
        \end{tabularx}
    \end{center}
    
    

    \subsubsection{FR-08: Delete Account}
    \begin{center}
        \begin{tabularx}{\textwidth}{|l|X|}
            \hline
            \textbf{ID} & UC-08 \\
            \hline
            \textbf{Name} & Delete Account \\
            \hline
            \textbf{Description} & User can delete the account if he/she can \\
            \hline
            \textbf{Actors} & Already User \\
            \hline
            \textbf{Assumptions} &  \\
            \hline
            \textbf{Triggers} & remove the user\_id from database by click on reset button \\
            \hline
            \textbf{Pre Conditions} & Already user database was be present in database \\
            \hline
            \textbf{Post Conditions} & No user of that user\_id will not be in database \\
            \hline
            \textbf{Main Course} & Account has been removed from database by the confirmation of user \\
            \hline
            \textbf{Alternative Course} & Error will be appeared by database \\
            \hline
            
        \end{tabularx}
    \end{center}
    \newpage
    

    \subsubsection{FR-09: Edit Account Info}
    \begin{center}
        \begin{tabularx}{\textwidth}{|l|X|}
            \hline
            \textbf{ID} & UC-09 \\
            \hline
            \textbf{Name} & Edit Account Info \\
            \hline
            \textbf{Description} & User can edit their account information \\
            \hline
            \textbf{Actors} & Already User \\
            \hline
            \textbf{Assumptions} &  \\
            \hline
            \textbf{Triggers} & Database of that user can be updated \\
            \hline
            \textbf{Pre Conditions} & Already data of that user will be in database  \\
            \hline
            \textbf{Post Conditions} & Update the new version of their information in database \\
            \hline
            \textbf{Main Course} & Information of the User can be updated if he/she can \\
            \hline
            \textbf{Alternative Course} & Error willbe displaced by the Database \\
            \hline
            
        \end{tabularx}
    \end{center}
    
    

    \subsubsection{FR-10: Edit Interface Preferences}
    \begin{center}
        \begin{tabularx}{\textwidth}{|l|X|}
            \hline
            \textbf{ID} & UC-10 \\
            \hline
            \textbf{Name} & Edit Interface Preferences \\
            \hline
            \textbf{Description} & User can edit their preferences of the UI of their dashboard \\
            \hline
            \textbf{Actors} & User \\
            \hline
            \textbf{Assumptions} &  \\
            \hline
            \textbf{Triggers} & User can change the primary and secondary Colors of the themes \\
            \hline
            \textbf{Pre Conditions} & Valid User \\
            \hline
            \textbf{Post Conditions} & UI preferences will be change according to User \\
            \hline
            \textbf{Main Course} & Interface preference will be changed by the user according to their preception  \\
            \hline
            \textbf{Alternative Course} & Error will be displayed it choose certain colors like black etc. \\
            \hline
            
        \end{tabularx}
    \end{center}
    
    

    \subsubsection{FR-11: Create Project}
    \begin{center}
        \begin{tabularx}{\textwidth}{|l|X|}
            \hline
            \textbf{ID} & UC-11 \\
            \hline
            \textbf{Name} & Create Project \\
            \hline
            \textbf{Description} & Manager can create a project which is stored in the database with the unique id \\
            \hline
            \textbf{Actors} & Manager \\
            \hline
            \textbf{Assumptions} &  \\
            \hline
            \textbf{Triggers} & The Project will created in the database with unique id \\
            \hline
            \textbf{Pre Conditions} & user will be valid and Project can't created before  \\
            \hline
            \textbf{Post Conditions} & Project is created by the Manager \\
            \hline
            \textbf{Main Course} & Project is created by the manager in database \\
            \hline
            \textbf{Alternative Course} & Error will be displayed of database  \\
            \hline
            
        \end{tabularx}
    \end{center}
    
    

    \subsubsection{FR-12: Edit Project}
    \begin{center}
        \begin{tabularx}{\textwidth}{|l|X|}
            \hline
            \textbf{ID} & UC-12 \\
            \hline
            \textbf{Name} & Edit Project \\
            \hline
            \textbf{Description} & The existing project will be edited by the manager if needed  \\
            \hline
            \textbf{Actors} & Manager \\
            \hline
            \textbf{Assumptions} &  \\
            \hline
            \textbf{Triggers} & The Information of the existing project will be updated in the database \\
            \hline
            \textbf{Pre Conditions} & Project is present already in database \\
            \hline
            \textbf{Post Conditions} & Project's Information will be updated by manager \\
            \hline
            \textbf{Main Course} & Manager can change the existing project if the requirements will be changed by the client \\
            \hline
            \textbf{Alternative Course} & Database's error will occurs \\
            \hline
            
        \end{tabularx}
    \end{center}
    \newpage
    

    \subsubsection{FR-13: Delete Project}
    \begin{center}
        \begin{tabularx}{\textwidth}{|l|X|}
            \hline
            \textbf{ID} & UC-13 \\
            \hline
            \textbf{Name} & Delete Project \\
            \hline
            \textbf{Description} & The Manager will delete the existing projects from the database \\
            \hline
            \textbf{Actors} & Managers \\
            \hline
            \textbf{Assumptions} &  \\
            \hline
            \textbf{Triggers} & Delete the existing project from the database by delete event \\
            \hline
            \textbf{Pre Conditions} & Project will be in the database which Manager will be deleted \\
            \hline
            \textbf{Post Conditions} & Project is deletedby the the manager \\
            \hline
            \textbf{Main Course} & The manager will delete the exiting project which is presnet in database \\
            \hline
            \textbf{Alternative Course} & Error will be displayed \\
            \hline
            
        \end{tabularx}
    \end{center}
    
    

    \subsubsection{FR-14: Add Developer}
    \begin{center}
        \begin{tabularx}{\textwidth}{|l|X|}
            \hline
            \textbf{ID} & UC-14 \\
            \hline
            \textbf{Name} & Add Developer \\
            \hline
            \textbf{Description} & The Manager will assign the developer of the company which is part of that Project \\
            \hline
            \textbf{Actors} & Managers \\
            \hline
            \textbf{Assumptions} &  \\
            \hline
            \textbf{Triggers} & Developer list will be shown to manager and manager will add the relevant developer \\
            \hline
            \textbf{Pre Conditions} & Developer and project will be added \\
            \hline
            \textbf{Post Conditions} & Developer will be added in the specific project \\
            \hline
            \textbf{Main Course} & To view the new upcoming project and take the decision of developer in project, Manager will added the developer in the project \\
            \hline
            \textbf{Alternative Course} & Error will be displayed \\
            \hline
            
        \end{tabularx}
    \end{center}
    
    

    \subsubsection{FR-15: Remove Developer}
    \begin{center}
        \begin{tabularx}{\textwidth}{|l|X|}
            \hline
            \textbf{ID} & UC-15 \\
            \hline
            \textbf{Name} & Remove Developer \\
            \hline
            \textbf{Description} & The manager can remove the developer after developer performs their duties \\
            \hline
            \textbf{Actors} & Manager \\
            \hline
            \textbf{Assumptions} &  \\
            \hline
            \textbf{Triggers} & Manager will remove the developer in the project anytime. \\
            \hline
            \textbf{Pre Conditions} & Developer will be added in that project \\
            \hline
            \textbf{Post Conditions} & developer is no more the part of the project \\
            \hline
            \textbf{Main Course} & Manager can manage the availabilty of the developer whenever he enter in the project or when he exits. \\
            \hline
            \textbf{Alternative Course} & Error will be displayed \\
            \hline
            
        \end{tabularx}
    \end{center}
    
    

    \subsubsection{FR-16: Edit Developer Permission}
    \begin{center}
        \begin{tabularx}{\textwidth}{|l|X|}
            \hline
            \textbf{ID} & UC-16 \\
            \hline
            \textbf{Name} & Edit Developer Permission \\
            \hline
            \textbf{Description} & The Manager can edit the role of the developer in the project \\
            \hline
            \textbf{Actors} & Managers \\
            \hline
            \textbf{Assumptions} &  \\
            \hline
            \textbf{Triggers} & Manager can edit the preference of the developer \\
            \hline
            \textbf{Pre Conditions} & Developer must have the part of the project \\
            \hline
            \textbf{Post Conditions} & Developer's Role will changed by the manager \\
            \hline
            \textbf{Main Course} & The Developer can change the role of the developer as he needs him in the project \\
            \hline
            \textbf{Alternative Course} & Error will be dislpayed by the database \\
            \hline
            
        \end{tabularx}
    \end{center}
    \newpage
    

    \subsubsection{FR-17: Calculate UCP}
    \begin{center}
        \begin{tabularx}{\textwidth}{|l|X|}
            \hline
            \textbf{ID} & UC-17 \\
            \hline
            \textbf{Name} & Calculate UCP \\
            \hline
            \textbf{Description} & Calculate the effort estimation of project via mean of UCP method \\
            \hline
            \textbf{Actors} & Manager , Developers \\
            \hline
            \textbf{Assumptions} &  \\
            \hline
            \textbf{Triggers} & The effort will calculted by UCP formula \\
            \hline
            \textbf{Pre Conditions} & The Project Use cases' information are inserted already \\
            \hline
            \textbf{Post Conditions} & Calculated Effort are given by UCP method  \\
            \hline
            \textbf{Main Course} & To find the Effort Estimation by UCP method of the entire project by the help of their use cases \\
            \hline
            \textbf{Alternative Course} & Error will displayed \\
            \hline
            
        \end{tabularx}
    \end{center}
    
    

    \subsubsection{FR-18: Get Machine Learning Estimation}
    \begin{center}
        \begin{tabularx}{\textwidth}{|l|X|}
            \hline
            \textbf{ID} & UC-18 \\
            \hline
            \textbf{Name} & Get Machine Learning Estimation \\
            \hline
            \textbf{Description} & Calculate the effort estimation of project via trained Machine Learning module \\
            \hline
            \textbf{Actors} & Manager  \\
            \hline
            \textbf{Assumptions} &  \\
            \hline
            \textbf{Triggers} & The effort will calculted by UCP formula \\
            \hline
            \textbf{Pre Conditions} & The Project Use cases' information are inserted already \\
            \hline
            \textbf{Post Conditions} & Calculated Effort are given by UCP method  \\
            \hline
            \textbf{Main Course} & To find the Effort Estimation by UCP method of the entire project by the help of their use cases \\
            \hline
            \textbf{Alternative Course} & Error will displayed \\
            \hline
            
        \end{tabularx}
    \end{center}
    
    

    \subsubsection{FR-19: Manual Estimate Round}
    \begin{center}
        \begin{tabularx}{\textwidth}{|l|X|}
            \hline
            \textbf{ID} & UC-19 \\
            \hline
            \textbf{Name} & Manual Estimate Round \\
            \hline
            \textbf{Description} & Calculate the effort estimation of project via Manual Estimation where the developer and experts will estimate the project while round \\
            \hline
            \textbf{Actors} & developers, manager \\
            \hline
            \textbf{Assumptions} &  \\
            \hline
            \textbf{Triggers} & The Effort will calculated by Manual technique \\
            \hline
            \textbf{Pre Conditions} & Developers were added by manager  in the project \\
            \hline
            \textbf{Post Conditions} & The effort will calculated after the many rounds of the experts' discussions  \\
            \hline
            \textbf{Main Course} & To find the Effort estimation by Manual technique by many rounds \\
            \hline
            \textbf{Alternative Course} & Error will displayed \\
            \hline
            
        \end{tabularx}
    \end{center}
    
    

    \subsubsection{FR-20: Estimation Bar}
    \begin{center}
        \begin{tabularx}{\textwidth}{|l|X|}
            \hline
            \textbf{ID} & UC-20 \\
            \hline
            \textbf{Name} & Estimation Bar \\
            \hline
            \textbf{Description} & All Estimation will be shown in the form of graphs \\
            \hline
            \textbf{Actors} & Manager , Developer \\
            \hline
            \textbf{Assumptions} &  \\
            \hline
            \textbf{Triggers} & The Estimate will be generated by the experts' round discussion \\
            \hline
            \textbf{Pre Conditions} & The estimation will be measured before that step  \\
            \hline
            \textbf{Post Conditions} & The charts will be appeared \\
            \hline
            \textbf{Main Course} & Estimation will be deliver in bar that anyone will be see and clearly judge the estimation of the project \\
            \hline
            \textbf{Alternative Course} & Error wil be displayed \\
            \hline
            
        \end{tabularx}
    \end{center}
    \newpage
    


\subsection{Agregated Use Case Diagram}
\begin{figure}[H]
    \includegraphics[height=23cm, width=0.8\textwidth]{./diagrams/Use Case/agrregated new.png}
    \centering 
    \caption{Agregated Use Case Diagram}
    \label{figurea}
    \end{figure}
    

% Chapter 4 Design

\section{Design}

\vspace{20mm}



\begin{abstract}
	
    This chapter is dedicated to representing the design of the system through a variety of different UML diagrams.
    
    
\end{abstract}

\vspace{20mm}

\large{\textbf{Outline}}

\begin{center}
    \begin{itemize}
        \item Architecture Diagram
        \item Entity Relationship Diagram
        \item Data Dictionary Diagram
        \item Data Flow Diagram
        \item Activity Diagram
        \item Sequence Diagram
        \item Collaboration Diagram
        \item State Transition Diagram
        \item Component Diagram
        \item Deployment Diagram
    \end{itemize}
\end{center}
\pagebreak

\subsection{Sequence Diagram}


\begin{figure}[H]
    \centering
    \caption{Sequence Diagram 1}
    \includegraphics[scale=0.5]{./diagrams/sequence/seq-01.png}
    \label{fig:seq-01}
    
\end{figure}


\begin{figure}[H]
    \centering
    \caption{Sequence Diagram 2}
    \includegraphics[scale=0.5]{./diagrams/sequence/seq-02.png}
    \label{fig:seq-02}
    
\end{figure}


\begin{figure}[H]
    \centering
    \caption{Sequence Diagram 3}
    \includegraphics[scale=0.5]{./diagrams/sequence/seq-03.png}
    \label{fig:seq-03}
    
\end{figure}


\begin{figure}[H]
    \centering
    \caption{Sequence Diagram 4}
    \includegraphics[scale=0.5]{./diagrams/sequence/seq-04.png}
    \label{fig:seq-04}
    
\end{figure}


\begin{figure}[H]
    \centering
    \caption{Sequence Diagram 5}
    \includegraphics[scale=0.5]{./diagrams/sequence/seq-05.png}
    \label{fig:seq-05}
    
\end{figure}


\begin{figure}[H]
    \centering
    \caption{Sequence Diagram 6}
    \includegraphics[scale=0.5]{./diagrams/sequence/seq-06.png}
    \label{fig:seq-06}
    
\end{figure}


\begin{figure}[H]
    \centering
    \caption{Sequence Diagram 7}
    \includegraphics[scale=0.5]{./diagrams/sequence/seq-07.png}
    \label{fig:seq-07}
    
\end{figure}


\begin{figure}[H]
    \centering
    \caption{Sequence Diagram 8}
    \includegraphics[scale=0.5]{./diagrams/sequence/seq-08.png}
    \label{fig:seq-08}
    
\end{figure}


\begin{figure}[H]
    \centering
    \caption{Sequence Diagram 9}
    \includegraphics[scale=0.5]{./diagrams/sequence/seq-09.png}
    \label{fig:seq-09}
    
\end{figure}


\begin{figure}[H]
    \centering
    \caption{Sequence Diagram 10}
    \includegraphics[scale=0.5]{./diagrams/sequence/seq-10.png}
    \label{fig:seq-10}
    
\end{figure}


\begin{figure}[H]
    \centering
    \caption{Sequence Diagram 11}
    \includegraphics[scale=0.5]{./diagrams/sequence/seq-11.png}
    \label{fig:seq-11}
    
\end{figure}


\begin{figure}[H]
    \centering
    \caption{Sequence Diagram 12}
    \includegraphics[scale=0.5]{./diagrams/sequence/seq-12.png}
    \label{fig:seq-12}
    
\end{figure}


\begin{figure}[H]
    \centering
    \caption{Sequence Diagram 13}
    \includegraphics[scale=0.5]{./diagrams/sequence/seq-13.png}
    \label{fig:seq-13}
    
\end{figure}


\begin{figure}[H]
    \centering
    \caption{Sequence Diagram 14}
    \includegraphics[scale=0.5]{./diagrams/sequence/seq-14.png}
    \label{fig:seq-14}
    
\end{figure}


\begin{figure}[H]
    \centering
    \caption{Sequence Diagram 15}
    \includegraphics[scale=0.5]{./diagrams/sequence/seq-15.png}
    \label{fig:seq-15}
    
\end{figure}


\begin{figure}[H]
    \centering
    \caption{Sequence Diagram 16}
    \includegraphics[scale=0.5]{./diagrams/sequence/seq-16.png}
    \label{fig:seq-16}
    
\end{figure}


\begin{figure}[H]
    \centering
    \caption{Sequence Diagram 17}
    \includegraphics[scale=0.5]{./diagrams/sequence/seq-17.png}
    \label{fig:seq-17}
    
\end{figure}


\begin{figure}[H]
    \centering
    \caption{Sequence Diagram 18}
    \includegraphics[scale=0.5]{./diagrams/sequence/seq-18.png}
    \label{fig:seq-18}
    
\end{figure}


\begin{figure}[H]
    \centering
    \caption{Sequence Diagram 19}
    \includegraphics[scale=0.5]{./diagrams/sequence/seq-19.png}
    \label{fig:seq-19}
    
\end{figure}



% Chapter 5 Testing

\section{Testing}

\vspace{20mm}


\Huge{\textbf{Testing}}

\vspace{20mm}


\begin{abstract}

    This chapter is dedicated to representing the Testing of the system
    through a variety of different testing techniques. These testing techniques will show
    various aspects of the system responses, including the relationships between the
    various entities, the relationships between the entities and the database responses,
    and the relationships between the entities and the user interface.
    The positive flow and development of states would also be demonstrated in the test cases.


\end{abstract}

\vspace{20mm}

\large{\textbf{Outline}}

\begin{center}
    \begin{itemize}
        \item Test Case Specification
        \item Black Box Testing
        \item Use Case Testing
        \item White Box Testing
        \item Performance Testing
        \item Load Testing
        \item Stress Testing
        \item Regression Testing
    \end{itemize}
\end{center}
\pagebreak


% Test Case Specification
\subsection{Test Case Specification}

% Black Box Testing
\subsection{ Black Box Testing}
Black Box testing is the Software testing method which is used to test the software without knowing the internal workings of the software.
Black box testing helps to find the gaps in functionality, usability and other features of the software. This form of testing is used to find the bugs in the software.
It improves software quality and reducesthe time to market. This form of testing mitigates the risk of software defects at the user's end.

Black-box testing attempts to find errors in the following categories:
\begin{itemize}
    \item Incorrect or missing functions.
    \item Interface errors.
    \item Errors in data structures or external database access.
    \item Behavior or performance errors.
    \item Initialization and termination errors.
\end{itemize}

Typical Black-box test design Techniques include:
\begin{itemize}
    \item Equivalence Partiotioning
    \item Boundary Value Analysis
    \item Decision Table
    \item State Transition Tables
    \item Use Case analysis
\end{itemize}


% Use Case Testing
\subsection{Use Case Testing}

% White Box Testing
\subsection{ White Box Testing}
In white-box testing an internal perspective of the system, as well as programming skills, are used to
design test cases. The tester chooses inputs to exercise paths through the code and determine the
expected outputs.

% Cyclomatic Testing

\subsubsection{ Cyclomatic Testing}
Cyclomatic complexity is a software metric used to indicate the complexity of a program. It is a
quantitative measure of the number of linearly independent paths through a program`s source code.

\begin{figure}[H]

    \centering
    \includegraphics[scale=0.7]{./diagrams/Activity Diagram/ad-01.png}
    \caption{Activity diagram of UC-1}
    \label{fig:act-01}

\end{figure}

\textbf{Cyclomatic Complexity}

M= E-N + 2(P)

E= number of edges

N= number of nodes

P= number of paths

E= 6,
N= 6,
P= 1,

M= 6-6+2(1)= 2

\begin{figure}[H]
    \centering
    \includegraphics[scale=0.7]{./diagrams/Activity Diagram/ad-02.png}
    \caption{Activity diagram of UC-2}
    \label{fig:act-02}

\end{figure}

\textbf{Cyclomatic Complexity}

M= E-N + 2(P)

E= number of edges

N= number of nodes

P= number of paths

E= 6,
N= 6,
P= 1,

M= 6-6+2(1)= 2

\begin{figure}[H]
    \centering
    \includegraphics[scale=0.7]{./diagrams/Activity Diagram/ad-03.png}
    \caption{Activity diagram of UC-3}
    \label{fig:act-03}

\end{figure}

\textbf{Cyclomatic Complexity}

M= E+N + 2(P)

E= number of edges

N= number of nodes

P= number of paths

E= 8,
N= 8,
P= 1,

M= 8-8+2(1)= 2

\begin{figure}[H]
    \centering
    \includegraphics[scale=0.7]{./diagrams/Activity Diagram/ad-04.png}
    \caption{Activity diagram of UC-4}
    \label{fig:act-04}

\end{figure}

\textbf{Cyclomatic Complexity}

M= E+N + 2(P)

E= number of edges

N= number of nodes

P= number of paths

E= 7,
N= 7,
P= 1,

M= 7-7+2(1)= 2

\begin{figure}[H]
    \centering
    \includegraphics[scale=0.7]{./diagrams/Activity Diagram/ad-05.png}
    \caption{Activity diagram of UC-7}
    \label{fig:act-05}

\end{figure}


\textbf{Cyclomatic Complexity}

M= E+N + 2(P)

E= number of edges

N= number of nodes

P= number of paths

E= 8,
N= 8,
P= 1,

M= 8-8+2(1)= 2

\begin{figure}[H]
    \centering
    \includegraphics[scale=0.6]{./diagrams/Activity Diagram/ad-06.png}
    \caption{Activity diagram of UC-6}
    \label{fig:act-06}

\end{figure}


\textbf{Cyclomatic Complexity}
\textbf{Cyclomatic Complexity}

M= E+N + 2(P)

E= number of edges

N= number of nodes

P= number of paths

E= 9,
N= 9,
P= 1,

M= 9-9+2(1)= 2

\begin{figure}[H]
    \centering
    \includegraphics[scale=0.6]{./diagrams/Activity Diagram/ad-07.png}
    \caption{Activity diagram of UC-7}
    \label{fig:act-07}

\end{figure}


\textbf{Cyclomatic Complexity}

M= E+N + 2(P)

E= number of edges

N= number of nodes

P= number of paths

E= 9,
N= 9,
P= 1,

M= 9-9+2(1)= 2

\begin{figure}[H]
    \centering
    \includegraphics[scale=0.6]{./diagrams/Activity Diagram/ad-08.png}
    \caption{Activity diagram of UC-8}
    \label{fig:act-08}

\end{figure}


\textbf{Cyclomatic Complexity}


M= E+N + 2(P)

E= number of edges

N= number of nodes

P= number of paths

E= 9,
N= 9,
P= 1,

M= 9-9+2(1)= 2

\begin{figure}[H]
    \centering
    \includegraphics[scale=0.6]{./diagrams/Activity Diagram/ad-09.png}
    \caption{Activity diagram of UC-9}
    \label{fig:act-09}

\end{figure}


\textbf{Cyclomatic Complexity}

M= E+N + 2(P)

E= number of edges

N= number of nodes

P= number of paths

E= 9,
N= 9,
P= 1,

M= 9-9+2(1)= 2

\begin{figure}[H]
    \centering
    \includegraphics[scale=0.7]{./diagrams/Activity Diagram/ad-10.png}
    \caption{Activity diagram of UC-10}
    \label{fig:act-10}

\end{figure}


\textbf{Cyclomatic Complexity}
M= E+N + 2(P)

E= number of edges

N= number of nodes

P= number of paths

E= 8,
N= 9,
P= 1,

M= 8-9+2(1)= 1

\begin{figure}[H]
    \centering
    \includegraphics[scale=0.7]{./diagrams/Activity Diagram/ad-11.png}
    \caption{Activity diagram of UC-11}
    \label{fig:act-11}

\end{figure}


\textbf{Cyclomatic Complexity}

M= E+N + 2(P)

E= number of edges

N= number of nodes

P= number of paths

E= 8,
N= 9,
P= 1,

M= 8-9+2(1)= 1

\begin{figure}[H]
    \centering
    \includegraphics[scale=0.7]{./diagrams/Activity Diagram/ad-12.png}
    \caption{Activity diagram of UC-12}
    \label{fig:act-12}

\end{figure}


\textbf{Cyclomatic Complexity}

M= E+N + 2(P)

E= number of edges

N= number of nodes

P= number of paths

E= 9,
N= 9,
P= 1,

M= 9-9+2(1)= 2

\begin{figure}[H]
    \centering
    \includegraphics[scale=0.7]{./diagrams/Activity Diagram/ad-13.png}
    \caption{Activity diagram of UC-13}
    \label{fig:act-13}

\end{figure}


\textbf{Cyclomatic Complexity}

M= E+N + 2(P)

E= number of edges

N= number of nodes

P= number of paths

E= 9,
N= 9,
P= 1,

M= 9-9+2(1)= 2

\begin{figure}[H]
    \centering
    \includegraphics[scale=0.7]{./diagrams/Activity Diagram/ad-14.png}
    \caption{Activity diagram of UC-14}
    \label{fig:act-14}

\end{figure}


\textbf{Cyclomatic Complexity}

M= E+N + 2(P)

E= number of edges

N= number of nodes

P= number of paths

E= 10,
N= 10,
P= 1,

M= 10-10+2(1)= 2

\begin{figure}[H]
    \centering
    \includegraphics[scale=0.7]{./diagrams/Activity Diagram/ad-15.png}
    \caption{Activity diagram of UC-15}
    \label{fig:act-15}

\end{figure}


\textbf{Cyclomatic Complexity}

M= E+N + 2(P)

E= number of edges

N= number of nodes

P= number of paths

E= 10,
N= 10,
P= 1,

M= 10-10+2(1)= 2

\begin{figure}[H]
    \centering
    \includegraphics[scale=0.7]{./diagrams/Activity Diagram/ad-16.png}
    \caption{Activity diagram of UC-16}
    \label{fig:act-16}

\end{figure}


\textbf{Cyclomatic Complexity}

M= E+N + 2(P)

E= number of edges

N= number of nodes

P= number of paths

E= 6,
N= 7,
P= 1,

M= 6-7+2(1)= 1

\begin{figure}[H]
    \centering
    \includegraphics[scale=0.7]{./diagrams/Activity Diagram/ad-17.png}
    \caption{Activity diagram of UC-17}
    \label{fig:act-17}

\end{figure}

\textbf{Cyclomatic Complexity}

M= E+N + 2(P)

E= number of edges

N= number of nodes

P= number of paths

E= 8,
N= 9,
P= 1,

M= 8-9+2(1)= 1

\begin{figure}[H]
    \centering
    \includegraphics[scale=0.7]{./diagrams/Activity Diagram/ad-18.png}
    \caption{Activity diagram of UC-18}
    \label{fig:act-18}

\end{figure}


\textbf{Cyclomatic Complexity}

M= E+N + 2(P)

E= number of edges

N= number of nodes

P= number of paths

E= 7,
N= 8,
P= 1,

M= 7-8+2(1)= 1

\begin{figure}[H]
    \centering
    \includegraphics[scale=0.7]{./diagrams/Activity Diagram/ad-19.png}
    \caption{Activity diagram of UC-19}
    \label{fig:act-19}

\end{figure}

\textbf{Cyclomatic Complexity}

M= E+N + 2(P)

E= number of edges

N= number of nodes

P= number of paths

E= 7,
N= 8,
P= 1,

M= 7-8+2(1)= 1

\begin{figure}[H]
    \centering
    \includegraphics[scale=0.7]{./diagrams/Activity Diagram/ad-20.png}
    \caption{Activity diagram of UC-20}
    \label{fig:act-20}

\end{figure}


\textbf{Cyclomatic Complexity}

M= E+N + 2(P)

E= number of edges

N= number of nodes

P= number of paths

E= 7,   N= 7,   P= 1,

M= 7-7+2(1)= 2


% Path Coverage
\subsubsection{ Path Coverage}
Path coverage tests all the paths of the program. This is a comprehensive technique which ensures
that all the paths of the program are traversed at least once.
\subsubsection{ Statement Coverage}
This technique requires every possible statement in the code to be tested at least once during the
testing process of software engineering.

\subsubsection{ Branch Coverage}
This technique checks every possible path (if-else and other conditional loops) of a software
application.

\subsection{Performance Testing}
Performance tests help to determine a system`s and application`s limitations, as well as the maximum
number of active users utilizing the application throughout servers. The Performance Test Plan and
Results is a combined document designed to more closely integrate performance test planning and
reporting.

\subsection{Load Testing}
Load testing is a technique used to determine the number of users that can be used to access a system.
The load testing plan is a document designed to more closely integrate load testing planning and
reporting.

\subsection{Stress Testing}
Stress Testing is a type of software testing that verifies stability and reliability of software application.
The goal of Stress testing is measuring software on its robustness and error handling capabilities
under extremely heavy load conditions and ensuring that software doesn`t crash under crunch
situations.
\subsection{Regression Testing}
Regression Testing is a type of software testing to confirm that a recent program or code change has
not adversely affected existing features.

% Chapter 6 Tool and Framework
%not Started

% Chapter 7 Summary and Conclusions
\section{Summary and Conclusions}

\vspace{20mm}


\Huge{\textbf{Summary and Conclusions}}

\vspace{20mm}


\begin{abstract}

    This chapter is dedicated to representing the Summary and Conclusions of the system.
    In first of this chapter we will sumarize the system and the system's features.
    In second of this chapter we will discuss about system's conclusions and outcomes.



\end{abstract}

\vspace{20mm}

\large{\textbf{Outline}}

\begin{center}
    \begin{itemize}
        \item Summary
        \item Conclusion
    \end{itemize}
\end{center}
\pagebreak


% Summary
\subsection{Summary}
The proposed system aspires to revolutionize and update the methodologies of software cost and effort
estimation. This particular area of software engineering hasn’t been worked on in order to create an
effective and usable system. The Proposed system implements state of the art machine learning algorithms
to give personalized and highly accurate estimations to all sorts of software projects. This accuracy is
achieved by training the model on modern datasets collected from actual software development companies
of successfully completed projects.This system will provide multiple methods to calculate estimates such
as the Delphi Technique, UCP Technique and Machine Learning techniques.
Along with the automated cost estimation services, the system will also provide a platform to collaborate
with developers and managers to discuss and agree on a better estimate. This platform is implemented to
enable the use of manual estimation techniques like Delphi.
The system can help software development companies save millions of dollars as well as countless hours
of time, spent on the cost and effort estimation process. This will enable faster development and better
software quality.
% Conclusion
\subsection{Conclusion}
At the end of the complete development of the project we have a web Application that can be used by
anyone to estimate the cost of any software project. The system is a used as a tool to estimate the
cost of any software project with three different methods which are Delphi Technique, UCP Technique and
Machine Learning techniques. If you want to use the system, you can signup the application from the
link and do the estimation.
Our Application is basically for Software Development Companies to estimate the cost of their projects.
The system is a used as a tool to estimate the cost of any software project by the companies's manager and manager
will create a project and add the developer to the project which will provide the esimation of the modules as well as
the whole system. The system will also provide a platform to collaborate with developers and managers to discuss and agree on a better estimate.
The system can help software development companies save millions of dollars as well as countless hours of time, spent on the cost and effort estimation process.
This will enable faster development and better software quality.

%not started

% Chapter 8  User Manual
\section{User Manual}

\vspace{20mm}


\Huge{\textbf{User Manual}}

\vspace{20mm}


\begin{abstract}

    This chapter is dedicated to representing the User Manual of the system
    through which new user can interact with the system. The complete user guide
    will be provided in the form of picture and user-guide.



\end{abstract}

\vspace{20mm}

\large{\textbf{Outline}}

\begin{center}
    \begin{itemize}
        \item Manager Functionalites
        \item Developer Functionalities
        \item Organization Functionalites
    \end{itemize}
\end{center}
\pagebreak


% Manager Functionalites
\subsection{Manager Functionalites}
% Developer Functionalites
\subsection{Developer Functionalites}
% Organization Functionalites
\subsection{Organization Functionalites}

%Incomplete


% Chapter 9  Lesson Learnt and Future Enhancements
\section{Lesson Learnt and Future Enhancements}

\vspace{20mm}


\Huge{\textbf{Lesson Learnt and Future Enhancements}}

\vspace{20mm}


\begin{abstract}

    In this chapter we will discuss the lessons learnt and future enhancements of the system.
    The lessons learnt will be discussed in the form of a list of problems and the future enhancements
    will be discussed in the form of a list of ideas.



\end{abstract}

\vspace{20mm}

\large{\textbf{Outline}}

\begin{center}
    \begin{itemize}
        \item Lesson Learnt
        \item Future Enhancements
    \end{itemize}
\end{center}
\pagebreak


% Lesson Learnt
\subsection{Lesson Learnt}

Everything in EffortAero, especially programming, takes longer than you expect.
Even if everything goes smoothly, it's difficult to predict how
long a feature will take. As we began working on it and determined
its specs and features, we realised that it was a significant project
and that we needed to begin working on it sooner. We learned how to
manage a project with a Two-person team, how to manage time, and
how to handle specifications.
Before you test the entire, test the components. Parts that have been
thoroughly tested save time. Integrating separate pieces can be difficult,
for example, due to mismatched or misunderstanding interfaces between modules.
It becomes much easier to trace down integration issues if you can trust
that the parts perform as planned.

% Future Enhancements
\subsection{Future Enhancements}
Firstly, we will deploy the system to a production environment.
We will then test the system in the production environment.
From the feedback we will determine what we should do to improve the system.
So then we will decide to add more functionality to the system such as:
\begin{itemize}
    \item Adding Manager as well as Developer to the system from outsource company.
    \item You will hire any developer you want like as Fiverr.
    \item You will hire any manager you want like as Fiverr.
    \item Adding NLP to determine the UCP if user can add the UCP in form of JPG or PNG.
    \item We will shift our Database as centralised database or distributed database.
\end{itemize}



\end{document}
