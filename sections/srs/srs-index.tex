\section{Software Requirements Specification}

\vspace{20mm}



\begin{abstract}
	In this chapter, We will discuss our
    functional requirements and non-functional
    requirements that will be used later on. There are two types of 
    requirements i.e Functional Requirement and Non-Functional Requirements.
    functional requirements define what the system does or must not do, non-functional
    requirements specify how the system should do it. Non-functional requirements do not
    affect the basic functionality of the system (hence the name, non-functional requirements).
    
\end{abstract}

\vspace{20mm}

\large{\textbf{Outline}}

\begin{center}
    \begin{itemize}
        \item Introduction
              \begin{itemize}
                  \item Purpose
                  \item Scope
              \end{itemize}
        \item Overall Description
        \item User Characteristics
        \item Specific Requirements
              \begin{itemize}
                  \item Functional Requirements
                  \item Non-Functional Requirements
              \end{itemize}
    \end{itemize}
\end{center}
\pagebreak


\subsection{Introduction}

    \subsubsection{Purpose}
    Here, is the purpose of our project is that the Cost Estimation is major component
    of planning a large scale software project.A wrong estimate of the project’s effort
    can lead to disastrous outcomes.Project managers can under estimate as well as 
    overestimate a project’s effort, so it is crucial to correctly estimate the 
    complexity of our project in order to deliver the product on time, provide a 
    much better developer experience and most importantly, don’t go over budget.


    \subsubsection{Scope}
    The major Scope of our Project is to estimate actual cost of the project and helpful to managers of the software industry are likely easy to calculate the budget of the developing software, and also ease to assigned the developer to the particular developing software.
    
    \subsubsection{Definition acronym \& abbrevation}
    The following acronym are used in our documents
    \begin{center}
        \begin{itemize}
            \item {\bfseries MERN:} MongoDB ExpressJS ReactJS NodeJS
            \item {\bfseries REST:} Restfull
            \item {\bfseries API:} Application Programming Interface
            \item {\bfseries DB:} Database
            \item {\bfseries SRS:} Software Requirements Specification
            \item {\bfseries UCP:} Use Case Point
            \item {\bfseries SPA:} Single Page Application
            \item {\bfseries CBR:} Content Base Reasoning
            \item {\bfseries FRs:} Functional Requirements
            \item {\bfseries NFRs:} Non-Functional Requirements
            \end{itemize}
    \end{center}
    
% continue from their
\subsection{Overall Description}
\subsubsection{Product Perspective}
Our Final Year Project is SPA(Single Page Application). To reduce the difficulties for those managers who feels 
difficulties to estimate the actual effort/cost of the upcoming project. Managers can also some other modules except
the Estimation i.e {\bfseries View/Update project} and {\bfseries Add/Remove Developers} in a project. Other than manager, there is a developer module in which developer 
can View and inform estimation of the project. Manager also have option to use our all methods for estimation as well as choose only one or two for calculating the actual estimate of the project.

    
\subsection{User Characteristics}
\blindtext[2]

\subsection{Specific Requirements}
\blindtext[2]

\subsubsection{Functional Requirements}
\newcommand{\splitcell}[2][c]{%
  \begin{tabular}[#1]{@{}c@{}}#2\end{tabular}}
\subsubsection{FR-01: Register}
\begin{center}
  \begin{tabularx}{\textwidth}{|l|X|}
      \hline
      \textbf{ID} & FR-01 \\
      \hline
      \textbf{Name} & Register \\
      \hline
      \textbf{Description} & Register a new user \\
      \hline
      \textbf{Input} & username, password, password\_confirmation, email \\
      \hline
      \textbf{Output} & A new user is created in the database \\
      \hline
      \textbf{Requirements} & Valid username, valid email, password that contains at least 8 characters, password\_confirmation that matches the password \\
      \hline
  \end{tabularx}
\end{center}

\subsubsection{FR-02: Login}
\begin{center}
  \begin{tabularx}{\textwidth}{|l|X|}
      \hline
      \textbf{ID} & FR-02 \\
      \hline
      \textbf{Name} & Login \\
      \hline
      \textbf{Description} & Login to the application \\
      \hline
      \textbf{Input} & username or email, password \\
      \hline
      \textbf{Output} & User is logged in, a JWT token is issued \\
      \hline
      \textbf{Requirements} & Valid username or email, matching password \\
      \hline
  \end{tabularx}
\end{center}

\subsubsection{FR-03: OAuth}
\begin{center}
  \begin{tabularx}{\textwidth}{|l|X|}
      \hline
      \textbf{ID} & FR-03 \\
      \hline
      \textbf{Name} & OAuth \\
      \hline
      \textbf{Description} & Login to the application using OAuth \\
      \hline
      \textbf{Input} & OAuth token \\
      \hline
      \textbf{Output} & User is logged in, a JWT token is issued \\
      \hline
      \textbf{Requirements} & Valid OAuth token \\
      \hline
  \end{tabularx}
\end{center}

\subsubsection{FR-04: Forget Password}
\begin{center}
  \begin{tabularx}{\textwidth}{|l|X|}
      \hline
      \textbf{ID} & FR-04 \\
      \hline
      \textbf{Name} & Forgot password \\
      \hline
      \textbf{Description} & Send an email to the user with a link to reset their password \\
      \hline
      \textbf{Input} & email \\
      \hline
      \textbf{Output} & An email is sent to the user with a link to reset their password \\
      \hline
      \textbf{Requirements} & Valid email \\
      \hline
  \end{tabularx}
\end{center}
\newpage

\subsubsection{FR-05: Reset Password}
\begin{center}
  \begin{tabularx}{\textwidth}{|l|X|}
      \hline
      \textbf{ID} & FR-05 \\
      \hline
      \textbf{Name} & Reset password \\
      \hline
      \textbf{Description} & Reset the user's password \\
      \hline
      \textbf{Input} & token, password, password\_confirmation \\
      \hline
      \textbf{Output} & The user's password is reset \\
      \hline
      \textbf{Requirements} & Valid token, valid password, password\_confirmation that matches the password \\
      \hline
  \end{tabularx}
\end{center}


\subsubsection{FR-06: Confirm Account}
\begin{center}
  \begin{tabularx}{\textwidth}{|l|X|}
      \hline
      \textbf{ID} & FR-06 \\
      \hline
      \textbf{Name} & Confirm account \\
      \hline
      \textbf{Description} & Confirm the user's account \\
      \hline
      \textbf{Input} & token \\
      \hline
      \textbf{Output} & The user's account is confirmed \\
      \hline
      \textbf{Requirements} & Valid token \\
      \hline
  \end{tabularx}
\end{center}

\subsubsection{FR-07: Delete Account}
\begin{center}
  \begin{tabularx}{\textwidth}{|l|X|}
      \hline
      \textbf{ID} & FR-07 \\
      \hline
      \textbf{Name} & Delete account \\
      \hline
      \textbf{Description} & Delete the user's account \\
      \hline
      \textbf{Input} & token \\
      \hline
      \textbf{Output} & The user's account is deleted \\
      \hline
      \textbf{Requirements} & Valid token \\
      \hline
  \end{tabularx}
\end{center}

\subsubsection{FR-08: Create Initial Account}
\begin{center}
  \begin{tabularx}{\textwidth}{|l|X|}
      \hline
      \textbf{ID} & FR-08 \\
      \hline
      \textbf{Name} & Create initial account \\
      \hline
      \textbf{Description} & Create the initial account \\
      \hline
      \textbf{Input} & role \\
      \hline
      \textbf{Output} & A user's information is complete \\
      \hline
      \textbf{Requirements} & A role, either developer or manager \\
      \hline
  \end{tabularx}
\end{center}

\subsubsection{FR-09: Create Initial Account usmanError}
\begin{center}
  \begin{tabularx}{\textwidth}{|l|X|}
      \hline
      \textbf{ID} & FR-09 \\
      \hline
      \textbf{Name} & Create initial account \\
      \hline
      \textbf{Description} & Create the initial account \\
      \hline
      \textbf{Input} & username, password, email \\
      \hline
      \textbf{Output} & The initial account is created \\
      \hline
      \textbf{Requirements} & Valid username, valid email, password that contains at least 8 characters \\
      \hline
  \end{tabularx}
\end{center}

\subsubsection{FR-10: Edit Account Info.}
\begin{center}
  \begin{tabularx}{\textwidth}{|l|X|}
      \hline
      \textbf{ID} & FR-10 \\
      \hline
      \textbf{Name} & Edit account info \\
      \hline
      \textbf{Description} & Edit the user's account info \\
      \hline
      \textbf{Input} & token, username, email \\
      \hline
      \textbf{Output} & The user's account info is updated \\
      \hline
      \textbf{Requirements} & Valid token, valid username, valid email \\
      \hline
  \end{tabularx}
\end{center}

\subsubsection{FR-11: Edit Interface Preference}
\begin{center}
  \begin{tabularx}{\textwidth}{|l|X|}
      \hline
      \textbf{ID} & FR-11 \\
      \hline
      \textbf{Name} & Edit interface preferences \\
      \hline
      \textbf{Description} & Edit the user's interface preferences \\
      \hline
      \textbf{Input} & token, theme, language \\
      \hline
      \textbf{Output} & The user's interface preferences are updated \\
      \hline
      \textbf{Requirements} & Valid token, valid theme, valid language \\
      \hline
  \end{tabularx}
\end{center}

\subsubsection{FR-12: Create Project}
\begin{center}
  \begin{tabularx}{\textwidth}{|l|X|}
      \hline
      \textbf{ID} & FR-12 \\
      \hline
      \textbf{Name} & Create Project \\
      \hline
      \textbf{Description} & The manager creates a fresh project \\
      \hline
      \textbf{Input} & use cases (file/forms), add developers(optional), project name, technology \\
      \hline
      \textbf{Output} & A new project is created in the database with the ID of its manager \\
      \hline
      \textbf{Requirements} & Use cases, name, technology name \\
      \hline
  \end{tabularx}
\end{center}

\subsubsection{FR-13: Edit Project}
\begin{center}
  \begin{tabularx}{\textwidth}{|l|X|}
      \hline
      \textbf{ID} & FR-13 \\
      \hline
      \textbf{Name} & Edit Project \\
      \hline
      \textbf{Description} & The manager edits an existing project \\
      \hline
      \textbf{Input} & use cases (file/forms), project name, technology, (all optional), project ID(needed) \\
      \hline
      \textbf{Output} & The referred project is  \\
      \hline
      \textbf{Requirements} & Use cases, name, technology name, manager logged in, project exists \\
      \hline
  \end{tabularx}
\end{center}

\subsubsection{FR-14: Delete Project}
\begin{center}
  \begin{tabularx}{\textwidth}{|l|X|}
      \hline
      \textbf{ID} & FR-14 \\
      \hline
      \textbf{Name} & Delete Project \\
      \hline
      \textbf{Description} & The manager removes an existing project from database \\
      \hline
      \textbf{Input} & Manager clicks on delete project and clicks Yes \\
      \hline
      \textbf{Output} & The referred project is removed from database \\
      \hline
      \textbf{Requirements} & Project ID, manager logged in \\
      \hline
  \end{tabularx}
\end{center}

\subsubsection{FR-15: Add Developer}
\begin{center}
  \begin{tabularx}{\textwidth}{|l|X|}
      \hline
      \textbf{ID} & FR-15 \\
      \hline
      \textbf{Name} & Add Developer \\
      \hline
      \textbf{Description} & The manager assigns a developer to the project \\
      \hline
      \textbf{Input} & Manager clicks on add new developer and types developer's username \\
      \hline
      \textbf{Output} & A new developer is added to projects developer's list \\
      \hline
      \textbf{Requirements} & Project ID, manager logged in, developer username \\
      \hline
  \end{tabularx}
\end{center}

\subsubsection{FR-16: Remove Developer}
\begin{center}
  \begin{tabularx}{\textwidth}{|l|X|}
      \hline
      \textbf{ID} & FR-16 \\
      \hline
      \textbf{Name} & Remove Developer \\
      \hline
      \textbf{Description} & The manager removes a developer from the project \\
      \hline
      \textbf{Input} & Manager goes into developers list and clicks remove button and then clicks Yes \\
      \hline
      \textbf{Output} & The developer is removed from the project's developers list \\
      \hline
      \textbf{Requirements} & Project ID, manager logged in, developer username \\
      \hline
  \end{tabularx}
\end{center}

\subsubsection{FR-17: Edit Developer Permission}
\begin{center}
  \begin{tabularx}{\textwidth}{|l|X|}
      \hline
      \textbf{ID} & FR-17 \\
      \hline
      \textbf{Name} & Edit Developer Permissions \\
      \hline
      \textbf{Description} & Manager changes a developer's access level to the project \\
      \hline
      \textbf{Input} & Manager goes tp developers list and clicks on change role on a developer. They then choose either View or Edit \\
      \hline
      \textbf{Output} & Project permissions are edited \\
      \hline
      \textbf{Requirements} & Project ID, Developer ID, Permission type, manager logged in \\
      \hline
  \end{tabularx}
\end{center}

\subsubsection{FR-18: Calculate UCP}
\begin{center}
  \begin{tabularx}{\textwidth}{|l|X|}
      \hline
      \textbf{ID} & FR-18 \\
      \hline
      \textbf{Name} & Calculate UCP \\
      \hline
      \textbf{Description} & Get UCP estimation of project using use cases \\
      \hline
      \textbf{Input} & Use Cases, environmental factors, technical factors \\
      \hline
      \textbf{Output} & An integer representing the UCP estimation in man hours \\
      \hline
      \textbf{Requirements} & manager logged in, use at least 2 cases are entered \\
      \hline
  \end{tabularx}
\end{center}

\subsubsection{FR-19: Get Machine Learning Estimate}
\begin{center}
  \begin{tabularx}{\textwidth}{|l|X|}
      \hline
      \textbf{ID} & FR-19 \\
      \hline
      \textbf{Name} & Get Machine Learning Estimate \\
      \hline
      \textbf{Description} & Calculate estimate using machine learning model \\
      \hline
      \textbf{Input} & Use Cases, environmental factors, technical factors \\
      \hline
      \textbf{Output} & An integer representing the machine learning estimation in man hours \\
      \hline
      \textbf{Requirements} & manager logged in, use at least 2 cases are entered \\
      \hline
  \end{tabularx}
\end{center}

\subsubsection{FR-20: Start Manual Estimation Round}
\begin{center}
  \begin{tabularx}{\textwidth}{|l|X|}
      \hline
      \textbf{ID} & FR-20 \\
      \hline
      \textbf{Name} & Start Manual Estimation Round \\
      \hline
      \textbf{Description} & Create an empty round in current project \\
      \hline
      \textbf{Input} & Manager clicks on start manual estimation \\
      \hline
      \textbf{Output} & Empty round is created in database for current project \\
      \hline
      \textbf{Requirements} & Manager logged in, project exists, project has at least two use case, at least one developer assigned to project \\
      \hline
  \end{tabularx}
\end{center}

\subsubsection{FR-21: End Round}
\begin{center}
  \begin{tabularx}{\textwidth}{|l|X|}
      \hline
      \textbf{ID} & FR-21 \\
      \hline
      \textbf{Name} & End Round \\
      \hline
      \textbf{Description} & End current round, locking in all estimates given so far \\
      \hline
      \textbf{Input} & Manager clicks end round \\
      \hline
      \textbf{Output} & Round length is increased by one and all estimates are locked in \\
      \hline
      \textbf{Requirements} & Round exists, manager logged in, at least one estimate is made \\
      \hline
  \end{tabularx}
\end{center}

\subsubsection{FR-22: Clear Round}
\begin{center}
  \begin{tabularx}{\textwidth}{|l|X|}
      \hline
      \textbf{ID} & FR-22 \\
      \hline
      \textbf{Name} & Clear Round \\
      \hline
      \textbf{Description} & Clear current round's estimates \\
      \hline
      \textbf{Input} & Manager clicks on clear round \\
      \hline
      \textbf{Output} & All of estimates made for this round are deleted \\
      \hline
      \textbf{Requirements} & Round exists, manager logged in, at least one estimate is made \\
      \hline
  \end{tabularx}
\end{center}

\subsubsection{FR-23: Calculate Round Average}
\begin{center}
  \begin{tabularx}{\textwidth}{|l|X|}
      \hline
      \textbf{ID} & FR-23 \\
      \hline
      \textbf{Name} & Calculate Round Average \\
      \hline
      \textbf{Description} & Process all given estimates and calculate average \\
      \hline
      \textbf{Input} & A round is finished \\
      \hline
      \textbf{Output} & Average is calculated and stored in database for current round \\
      \hline
      \textbf{Requirements} & Round exists, manager logged in, at least one estimate is made \\
      \hline
  \end{tabularx}
\end{center}

\subsubsection{FR-24: Finalize Manual Estimation}
\begin{center}
  \begin{tabularx}{\textwidth}{|l|X|}
      \hline
      \textbf{ID} & FR-24 \\
      \hline
      \textbf{Name} & Finalize Manual Estimation \\
      \hline
      \textbf{Description} & Multiple rounds are finished and the team has agreed on a final estimate \\
      \hline
      \textbf{Input} & Manager clicks on finalize button \\
      \hline
      \textbf{Output} & Manual estimation is calculated and stored in database for current round \\
      \hline
      \textbf{Requirements} & Al least one round finished, manager logged in \\
      \hline
  \end{tabularx}
\end{center}

\subsubsection{FR-25: AllEstimate Bar Chart}
\begin{center}
  \begin{tabularx}{\textwidth}{|l|X|}
      \hline
      \textbf{ID} & FR-25 \\
      \hline
      \textbf{Name} & All Estimate Bar Chart \\
      \hline
      \textbf{Description} & Generate a bar chart comparing all types of estimates \\
      \hline
      \textbf{Input} & UCP estimate, manual estimate, ML estimate \\
      \hline
      \textbf{Output} & A bar chart is generated \\
      \hline
      \textbf{Requirements} & At least one type of estimate calculated \\
      \hline
  \end{tabularx}
\end{center}

\subsubsection{FR-26: All Manual Estimate Rounds' Line Chart}
\begin{center}
  \begin{tabularx}{\textwidth}{|l|X|}
      \hline
      \textbf{ID} & FR-26 \\
      \hline
      \textbf{Name} & All Manual Estimate Rounds' Line Chart \\
      \hline
      \textbf{Description} & A line chart of all rounds' estimates generated overtime \\
      \hline
      \textbf{Input} & All rounds' averages \\
      \hline
      \textbf{Output} & A line chart is generated \\
      \hline
      \textbf{Requirements} & At least one round has been finished \\
      \hline
  \end{tabularx}
\end{center}

\subsubsection{FR-27: All Manual Estimate Rounds' Bar Chart usman ERROR}
\begin{center}
  \begin{tabularx}{\textwidth}{|l|X|}
      \hline
      \textbf{ID} & FR-27 \\
      \hline
      \textbf{Name} & All Manual Estimate Rounds' Bar Chart \\
      \hline
      \textbf{Description} & A bar chart with showing the progressing of manual estimation rounds while highlighting each developer's estimate \\
      \hline
      \textbf{Input} & round estimates , developers' estimates \\
      \hline
      \textbf{Output} & A bar chart is generated \\
      \hline
      \textbf{Requirements} & At least one round has been finished \\
      \hline
  \end{tabularx}
\end{center}

\subsubsection{FR-28: All Machine Learning Estimates Line Chart}
\begin{center}
  \begin{tabularx}{\textwidth}{|l|X|}
      \hline
      \textbf{ID} & FR-28 \\
      \hline
      \textbf{Name} & All Machine Learning Estimates Line Chart \\
      \hline
      \textbf{Description} & Generate a line chart of all machine learning estimates so far \\
      \hline
      \textbf{Input} & Machine learning estimates \\
      \hline
      \textbf{Output} & A line chart is generated \\
      \hline
      \textbf{Requirements} & At least one ML estimate calculated \\
      \hline
  \end{tabularx}
\end{center}

\subsubsection{FR-29: Pie Chart of Use Cases Weight}
\begin{center}
  \begin{tabularx}{\textwidth}{|l|X|}
      \hline
      \textbf{ID} & FR-29 \\
      \hline
      \textbf{Name} & Pie Chart of Use Cases Weight \\
      \hline
      \textbf{Description} & A pie chart showcasing the most important use cases \\
      \hline
      \textbf{Input} & Use cases \\
      \hline
      \textbf{Output} & A pie chart is generated \\
      \hline
      \textbf{Requirements} & At least two use cases are entered \\
      \hline
  \end{tabularx}
\end{center}

\subsubsection{FR-30: Template of FR}
\begin{center}
  \begin{tabularx}{\textwidth}{|l|X|}
      \hline
      \textbf{ID} & FR-30 \\
      \hline
      \textbf{Name} &  \\
      \hline
      \textbf{Description} &  \\
      \hline
      \textbf{Input} &  \\
      \hline
      \textbf{Output} &  \\
      \hline
      \textbf{Requirements} &  \\
      \hline
  \end{tabularx}
\end{center}

\subsection{Non Functional Requirements}


\newpage

% \begin{center}
%     \begin{table}[H]
%         \centering
%         \begin{tabular}{@{}|l|l|l|l|@{}}
%             \hline
%             ID                           & \multicolumn{3}{l|}{FR-01}                                                                \\ \midrule
%             Name                         & \multicolumn{3}{l|}{Admin Sign-Up}                                                        \\ \midrule
%             Description                  & Input                              & Output                   & Basic Workflow            \\ \midrule
%             Admin register their account & \begin{tabular}[c]{@{}l@{}}User name and password must be \\ greater than eight letters\\ Details of user\end{tabular}          & Creation of User account & \begin{tabular}[c]{@{}l@{}}Enter user details and  added\\ into the database records\end{tabular} \\ \bottomrule
%         \end{tabular}
%         \caption{Functional Requirement 01: Admin Sign-Up}
%         \label{table:FR01}
%     \end{table}
% \end{center}

