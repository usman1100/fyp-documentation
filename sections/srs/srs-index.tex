\section{Software Requirements Specification}

\vspace{20mm}



\begin{abstract}
	In this chapter, We will discuss our
    functional requirements and non-functional
    requirements that will be used later on. There are two types of 
    requirements i.e Functional Requirement and Non-Functional Requirements.
    functional requirements define what the system does or must not do, non-functional
    requirements specify how the system should do it. Non-functional requirements do not
    affect the basic functionality of the system (hence the name, non-functional requirements).
    
\end{abstract}

\vspace{20mm}

\large{\textbf{Outline}}

\begin{center}
    \begin{itemize}
        \item Introduction
              \begin{itemize}
                  \item Purpose
                  \item Scope
              \end{itemize}
        \item Overall Description
        \item User Characteristics
        \item Specific Requirements
              \begin{itemize}
                  \item Functional Requirements
                  \item Non-Functional Requirements
              \end{itemize}
    \end{itemize}
\end{center}
\pagebreak


\subsection{Introduction}

    \subsubsection{Purpose}
    Here, is the purpose of our project is that the Cost Estimation is major component
    of planning a large scale software project.A wrong estimate of the project’s effort
    can lead to disastrous outcomes.Project managers can under estimate as well as 
    overestimate a project’s effort, so it is crucial to correctly estimate the 
    complexity of our project in order to deliver the product on time, provide a 
    much better developer experience and most importantly, don’t go over budget.


    \subsubsection{Scope}
    The major Scope of our Project is to estimate actual cost of the project and helpful to managers of the software industry are likely easy to calculate the budget of the developing software, and also ease to assigned the developer to the particular developing software.
    
    \subsubsection{Definition acronym \& abbrevation}
    The following acronym are used in our documents
    \begin{center}
        \begin{itemize}
            \item {\bfseries MERN:} MongoDB ExpressJS ReactJS NodeJS
            \item {\bfseries REST:} Restfull
            \item {\bfseries API:} Application Programming Interface
            \item {\bfseries DB:} Database
            \item {\bfseries SRS:} Software Requirements Specification
            \item {\bfseries UCP:} Use Case Point
            \item {\bfseries SPA:} Single Page Application
            \item {\bfseries CBR:} Content Base Reasoning
            \item {\bfseries FRs:} Functional Requirements
            \item {\bfseries NFRs:} Non-Functional Requirements
            \end{itemize}
    \end{center}
    
% continue from their
\subsection{Overall Description}
\subsubsection{Product Perspective}
Our Final Year Project is SPA(Single Page Application). To reduce the difficulties for those managers who feels 
difficulties to estimate the actual effort/cost of the upcoming project. Managers can also some other modules except
the Estimation i.e {\bfseries View/Update project} and {\bfseries Add/Remove Developers} in a project. Other than manager, there is a developer module in which developer 
can View and inform estimation of the project. Manager also have option to use our all methods for estimation as well as choose only one or two for calculating the actual estimate of the project.

    \paragraph{System Interfaces:}
    Our Proposed system is built from the scratch from the modern technology and add one feature from the exciting application. 
    \paragraph{User interface: }
    The interface of the website is designed to be simple and interactive and be Single Page Application to improve the user experience. 
    For attracting the developer and managers, we have facilitate them to change the theme of the system as they can feel to easy for work.
    As we know who attract with our system has well knowledge of how to operate a website but we kept as easy to operatable.
    \paragraph{Prototype}
    {\huge Prototype lgani ha yahan}
    \paragraph{Hardware Interfaces}
    The website can be used on any device such as a mobile phone laptop, computer or a tab as long as it has active Internet connection.
    \paragraph{Software Interface}
    Software interfaces include operating system for the computer, a browser to access the website with internet connectivity.
\subsubsection{Product Funtions}
    Product Function are divided into three categories depending upon the type of user. Product Functions are follows:
    Manager , Developer , Software Association
    \\{\huge Functional Requirements table without description  or with description likhni wahan  \\ }
    \subsubsection{Functional Requirements of User}
    

%\subsubsection{Functional Requirements}
%\paragraph{Hello G}

\subsection{User Characteristics}
\subsection*{Application User}
After the compeletion of project, application user will able to estimate the budget as well as effort of the new project and later on it will be save for the future if any similar project will be arrive they make sure the effort of that type of software will be this.
Application User will easily entertain their client with the almost accurate budget. 
\subsubsection{Constraint}
The application need an internet access it will not work without internet access.
\subsubsection{Assumptions and dependencies}
\begin{center}
    \begin{itemize}
        \item This Application will target the Managers of the Software Associations to find the almost actual effort as well as budget of the upcoming project.
        \item User should know the basic knowledge of the system.
    \end{itemize}
\end{center}
\subsubsection{Non Functional Requirements}
\paragraph{Usability}
Usability of the system is high because it is easy to use. Names of choices are very clear and according to their functionality. Its interface is not difficult as user can understand
everything clearly and use its functionality without any confusion.  
\paragraph{Maintainability}
System will be developed module wise so Maintainability will be achieved by architecture.
\paragraph{Reusability}
The application designed so that its code can be reused in other application similar to this application and as well as in other applications.
\paragraph{Response Time}
Response time depends on the speed of internet of user. Higher the spped of internet higher will be the reponse time.
