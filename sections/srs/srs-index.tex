\section{Software Requirements Specification}

\vspace{20mm}



\begin{abstract}
	In this chapter, We will discuss our
    functional requirements and non-functional
    requirements that will be used later on. There are two types of 
    requirements i.e Functional Requirement and Non-Functional Requirements.
    functional requirements define what the system does or must not do, non-functional
    requirements specify how the system should do it. Non-functional requirements do not
    affect the basic functionality of the system (hence the name, non-functional requirements).
    
\end{abstract}

\vspace{20mm}

\large{\textbf{Outline}}

\begin{center}
    \begin{itemize}
        \item Introduction
              \begin{itemize}
                  \item Purpose
                  \item Scope
              \end{itemize}
        \item Overall Description
        \item User Characteristics
        \item Specific Requirements
              \begin{itemize}
                  \item Functional Requirements
                  \item Non-Functional Requirements
              \end{itemize}
    \end{itemize}
\end{center}
\pagebreak


\subsection{Introduction}

    \subsubsection{Purpose}
    Here, is the purpose of our project is that the Cost Estimation is major component
    of planning a large scale software project.A wrong estimate of the project’s effort
    can lead to disastrous outcomes.Project managers can under estimate as well as 
    overestimate a project’s effort, so it is crucial to correctly estimate the 
    complexity of our project in order to deliver the product on time, provide a 
    much better developer experience and most importantly, don’t go over budget.


    \subsubsection{Scope}
    The major Scope of our Project is to estimate actual cost of the project and helpful to managers of the software industry are likely easy to calculate the budget of the developing software, and also ease to assigned the developer to the particular developing software.
    
    \subsubsection{Definition acronym \& abbrevation}
    The following acronym are used in our documents
    \begin{center}
        \begin{itemize}
            \item {\bfseries MERN:} MongoDB ExpressJS ReactJS NodeJS
            \item {\bfseries REST:} Restfull
            \item {\bfseries API:} Application Programming Interface
            \item {\bfseries DB:} Database
            \item {\bfseries SRS:} Software Requirements Specification
            \item {\bfseries UCP:} Use Case Point
            \item {\bfseries SPA:} Single Page Application
            \item {\bfseries CBR:} Content Base Reasoning
            \item {\bfseries FRs:} Functional Requirements
            \item {\bfseries NFRs:} Non-Functional Requirements
            \end{itemize}
    \end{center}
    
% continue from their
\subsection{Overall Description}
\subsubsection{Product Perspective}
Our Final Year Project is SPA(Single Page Application). To reduce the difficulties for those managers who feels 
difficulties to estimate the actual effort/cost of the upcoming project. Managers can also some other modules except
the Estimation i.e {\bfseries View/Update project} and {\bfseries Add/Remove Developers} in a project. Other than manager, there is a developer module in which developer 
can View and inform estimation of the project. Manager also have option to use our all methods for estimation as well as choose only one or two for calculating the actual estimate of the project.

    
\subsection{User Characteristics}
\blindtext[2]

\subsection{Specific Requirements}
\blindtext[2]

\subsubsection{Functional Requirements}
\paragraph{Hello G}

\subsection{Functional Requirements of User}


\newpage

\begin{center}
    \begin{table}[H]
        \centering
        \begin{tabular}{@{}|l|l|l|l|@{}}
            \hline
            ID                           & \multicolumn{3}{l|}{FR-01}                                                                \\ \midrule
            Name                         & \multicolumn{3}{l|}{Admin Sign-Up}                                                        \\ \midrule
            Description                  & Input                              & Output                   & Basic Workflow            \\ \midrule
            Admin register their account & \begin{tabular}[c]{@{}l@{}}User name and password must be \\ greater than eight letters\\ Details of user\end{tabular}          & Creation of User account & \begin{tabular}[c]{@{}l@{}}Enter user details and  added\\ into the database records\end{tabular} \\ \bottomrule
        \end{tabular}
        \caption{Functional Requirement 01: Admin Sign-Up}
        \label{table:FR01}
    \end{table}
\end{center}

