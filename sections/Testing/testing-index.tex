
\section{Testing}

\vspace{20mm}


\Huge{\textbf{Testing}}

\vspace{20mm}


\begin{abstract}

    This chapter is dedicated to representing the Testing of the system
    through a variety of different testing techniques. These testing techniques will show
    various aspects of the system responses, including the relationships between the
    various entities, the relationships between the entities and the database responses,
    and the relationships between the entities and the user interface.
    The positive flow and development of states would also be demonstrated in the test cases.


\end{abstract}

\vspace{20mm}

\large{\textbf{Outline}}

\begin{center}
    \begin{itemize}
        \item Test Case Specification
        \item Black Box Testing
        \item Use Case Testing
        \item White Box Testing
        \item Performance Testing
        \item Load Testing
        \item Stress Testing
        \item Regression Testing
    \end{itemize}
\end{center}
\pagebreak


% Test Case Specification
\subsection{Test Case Specification}

% Black Box Testing
\subsection{ Black Box Testing}

% Use Case Testing
\subsection{Use Case Testing}

% White Box Testing
\subsection{ White Box Testing}
In white-box testing an internal perspective of the system, as well as programming skills, are used to
design test cases. The tester chooses inputs to exercise paths through the code and determine the
expected outputs.


% Cyclomatic Testing

\subsubsection{ Cyclomatic Testing}
Cyclomatic complexity is a software metric used to indicate the complexity of a program. It is a
quantitative measure of the number of linearly independent paths through a program`s source code.

\begin{figure}[H]

    \centering
    \includegraphics[scale=0.7]{./diagrams/Activity Diagram/ad-01.png}
    \caption{Activity diagram of UC-1}
    \label{fig:act-01}

\end{figure}

\textbf{Cyclomatic Complexity}

M= E-N + 2(P)

E= number of edges

N= number of nodes

P= number of paths

E= 6,
N= 6,
P= 1,

M= 6-6+2(1)= 2

\begin{figure}[H]
    \centering
    \includegraphics[scale=0.7]{./diagrams/Activity Diagram/ad-02.png}
    \caption{Activity diagram of UC-2}
    \label{fig:act-02}

\end{figure}

\textbf{Cyclomatic Complexity}

M= E-N + 2(P)

E= number of edges

N= number of nodes

P= number of paths

E= 6,
N= 6,
P= 1,

M= 6-6+2(1)= 2

\begin{figure}[H]
    \centering
    \includegraphics[scale=0.7]{./diagrams/Activity Diagram/ad-03.png}
    \caption{Activity diagram of UC-3}
    \label{fig:act-03}

\end{figure}

\textbf{Cyclomatic Complexity}

M= E+N + 2(P)

E= number of edges

N= number of nodes

P= number of paths

E= 8,
N= 8,
P= 1,

M= 8-8+2(1)= 2

\begin{figure}[H]
    \centering
    \includegraphics[scale=0.7]{./diagrams/Activity Diagram/ad-04.png}
    \caption{Activity diagram of UC-4}
    \label{fig:act-04}

\end{figure}

\textbf{Cyclomatic Complexity}

M= E+N + 2(P)

E= number of edges

N= number of nodes

P= number of paths

E= 7,
N= 7,
P= 1,

M= 7-7+2(1)= 2

\begin{figure}[H]
    \centering
    \includegraphics[scale=0.7]{./diagrams/Activity Diagram/ad-05.png}
    \caption{Activity diagram of UC-7}
    \label{fig:act-05}

\end{figure}


\textbf{Cyclomatic Complexity}

M= E+N + 2(P)

E= number of edges

N= number of nodes

P= number of paths

E= 8,
N= 8,
P= 1,

M= 8-8+2(1)= 2

\begin{figure}[H]
    \centering
    \includegraphics[scale=0.6]{./diagrams/Activity Diagram/ad-06.png}
    \caption{Activity diagram of UC-6}
    \label{fig:act-06}

\end{figure}


\textbf{Cyclomatic Complexity}
\textbf{Cyclomatic Complexity}

M= E+N + 2(P)

E= number of edges

N= number of nodes

P= number of paths

E= 9,
N= 9,
P= 1,

M= 9-9+2(1)= 2

\begin{figure}[H]
    \centering
    \includegraphics[scale=0.6]{./diagrams/Activity Diagram/ad-07.png}
    \caption{Activity diagram of UC-7}
    \label{fig:act-07}

\end{figure}


\textbf{Cyclomatic Complexity}

M= E+N + 2(P)

E= number of edges

N= number of nodes

P= number of paths

E= 9,
N= 9,
P= 1,

M= 9-9+2(1)= 2

\begin{figure}[H]
    \centering
    \includegraphics[scale=0.6]{./diagrams/Activity Diagram/ad-08.png}
    \caption{Activity diagram of UC-8}
    \label{fig:act-08}

\end{figure}


\textbf{Cyclomatic Complexity}


M= E+N + 2(P)

E= number of edges

N= number of nodes

P= number of paths

E= 9,
N= 9,
P= 1,

M= 9-9+2(1)= 2

\begin{figure}[H]
    \centering
    \includegraphics[scale=0.6]{./diagrams/Activity Diagram/ad-09.png}
    \caption{Activity diagram of UC-9}
    \label{fig:act-09}

\end{figure}


\textbf{Cyclomatic Complexity}

M= E+N + 2(P)

E= number of edges

N= number of nodes

P= number of paths

E= 9,
N= 9,
P= 1,

M= 9-9+2(1)= 2

\begin{figure}[H]
    \centering
    \includegraphics[scale=0.7]{./diagrams/Activity Diagram/ad-10.png}
    \caption{Activity diagram of UC-10}
    \label{fig:act-10}

\end{figure}


\textbf{Cyclomatic Complexity}
M= E+N + 2(P)

E= number of edges

N= number of nodes

P= number of paths

E= 8,
N= 9,
P= 1,

M= 8-9+2(1)= 1

\begin{figure}[H]
    \centering
    \includegraphics[scale=0.7]{./diagrams/Activity Diagram/ad-11.png}
    \caption{Activity diagram of UC-11}
    \label{fig:act-11}

\end{figure}


\textbf{Cyclomatic Complexity}

M= E+N + 2(P)

E= number of edges

N= number of nodes

P= number of paths

E= 8,
N= 9,
P= 1,

M= 8-9+2(1)= 1

\begin{figure}[H]
    \centering
    \includegraphics[scale=0.7]{./diagrams/Activity Diagram/ad-12.png}
    \caption{Activity diagram of UC-12}
    \label{fig:act-12}

\end{figure}


\textbf{Cyclomatic Complexity}

M= E+N + 2(P)

E= number of edges

N= number of nodes

P= number of paths

E= 9,
N= 9,
P= 1,

M= 9-9+2(1)= 2

\begin{figure}[H]
    \centering
    \includegraphics[scale=0.7]{./diagrams/Activity Diagram/ad-13.png}
    \caption{Activity diagram of UC-13}
    \label{fig:act-13}

\end{figure}


\textbf{Cyclomatic Complexity}

M= E+N + 2(P)

E= number of edges

N= number of nodes

P= number of paths

E= 9,
N= 9,
P= 1,

M= 9-9+2(1)= 2

\begin{figure}[H]
    \centering
    \includegraphics[scale=0.7]{./diagrams/Activity Diagram/ad-14.png}
    \caption{Activity diagram of UC-14}
    \label{fig:act-14}

\end{figure}


\textbf{Cyclomatic Complexity}

M= E+N + 2(P)

E= number of edges

N= number of nodes

P= number of paths

E= 10,
N= 10,
P= 1,

M= 10-10+2(1)= 2

\begin{figure}[H]
    \centering
    \includegraphics[scale=0.7]{./diagrams/Activity Diagram/ad-15.png}
    \caption{Activity diagram of UC-15}
    \label{fig:act-15}

\end{figure}


\textbf{Cyclomatic Complexity}

M= E+N + 2(P)

E= number of edges

N= number of nodes

P= number of paths

E= 10,
N= 10,
P= 1,

M= 10-10+2(1)= 2

\begin{figure}[H]
    \centering
    \includegraphics[scale=0.7]{./diagrams/Activity Diagram/ad-16.png}
    \caption{Activity diagram of UC-16}
    \label{fig:act-16}

\end{figure}


\textbf{Cyclomatic Complexity}

M= E+N + 2(P)

E= number of edges

N= number of nodes

P= number of paths

E= 6,
N= 7,
P= 1,

M= 6-7+2(1)= 1

\begin{figure}[H]
    \centering
    \includegraphics[scale=0.7]{./diagrams/Activity Diagram/ad-17.png}
    \caption{Activity diagram of UC-17}
    \label{fig:act-17}

\end{figure}

\textbf{Cyclomatic Complexity}

M= E+N + 2(P)

E= number of edges

N= number of nodes

P= number of paths

E= 8,
N= 9,
P= 1,

M= 8-9+2(1)= 1

\begin{figure}[H]
    \centering
    \includegraphics[scale=0.7]{./diagrams/Activity Diagram/ad-18.png}
    \caption{Activity diagram of UC-18}
    \label{fig:act-18}

\end{figure}


\textbf{Cyclomatic Complexity}

M= E+N + 2(P)

E= number of edges

N= number of nodes

P= number of paths

E= 7,
N= 8,
P= 1,

M= 7-8+2(1)= 1

\begin{figure}[H]
    \centering
    \includegraphics[scale=0.7]{./diagrams/Activity Diagram/ad-19.png}
    \caption{Activity diagram of UC-19}
    \label{fig:act-19}

\end{figure}

\textbf{Cyclomatic Complexity}

M= E+N + 2(P)

E= number of edges

N= number of nodes

P= number of paths

E= 7,
N= 8,
P= 1,

M= 7-8+2(1)= 1

\begin{figure}[H]
    \centering
    \includegraphics[scale=0.7]{./diagrams/Activity Diagram/ad-20.png}
    \caption{Activity diagram of UC-20}
    \label{fig:act-20}

\end{figure}


\textbf{Cyclomatic Complexity}

M= E+N + 2(P)

E= number of edges

N= number of nodes

P= number of paths

E= 7,   N= 7,   P= 1,

M= 7-7+2(1)= 2


% Path Coverage
\subsubsection{ Path Coverage}
Path coverage tests all the paths of the program. This is a comprehensive technique which ensures
that all the paths of the program are traversed at least once.
\subsubsection{ Statement Coverage}
This technique requires every possible statement in the code to be tested at least once during the
testing process of software engineering.

\subsubsection{ Branch Coverage}
This technique checks every possible path (if-else and other conditional loops) of a software
application.

\subsection{Performance Testing}
Performance tests help to determine a system`s and application`s limitations, as well as the maximum
number of active users utilizing the application throughout servers. The Performance Test Plan and
Results is a combined document designed to more closely integrate performance test planning and
reporting.

\subsection{Load Testing}
Load testing is a technique used to determine the number of users that can be used to access a system.
The load testing plan is a document designed to more closely integrate load testing planning and
reporting.

\subsection{Stress Testing}
Stress Testing is a type of software testing that verifies stability and reliability of software application.
The goal of Stress testing is measuring software on its robustness and error handling capabilities
under extremely heavy load conditions and ensuring that software doesn`t crash under crunch
situations.
\subsection{Regression Testing}
Regression Testing is a type of software testing to confirm that a recent program or code change has
not adversely affected existing features.