
\subsection{Tools}

There are multiple types / categories of tools that we use to develop our system. Theses categoreies include development, design and testing tools, libraries and frameworks.

\subsubsection{Design Tools}
The design phase consisted of the following activities: Designing rough prototypes of the system and creating diagrams for explaining different phases of the system and it's workings.

\hspace{5mm}

\textbf{Figma}

A very well known design and prototype used by millions of professional designers. The prototype for the proposed system was developed using this tool. Figma is also used for the development of some of the assets used in the system such as logos and banners.

\hspace{5mm}

\textbf{Draw.io}
Draw.io is a free and open source software that is used to create and edit diagrams. It is used to create the diagrams for the proposed system. This tool is a very versitile peice of software, it is cross platforrm, can run in a browser as well a standalone application.


\subsubsection{Development Tools}

\hspace{5mm}


\textbf{Visual Studio Code}


Visual Studio Code or sometimes shortened as VSCode is a free and open source software that is used to develop and edit code. It features a rich eco systems consisting of community driven extenions, themes, icon packs, linting systems, an integrated debugger, built-in terminal and so much more.



We chose this tool considering it's massive popularity and the community surrounding it. Since VSCode is an open source project, new issues are raised and resolved by the community on regular basis.


Along that, it is also cross-platform. Since us team members work on multiple platoforms that inlucde Linux, Mac OS and Windows, VSCode is a great tool for us to use.

\hspace{5mm}


\textbf{MongoDB Compass}


MongoDB Compass dubbed as the GUI for MongoDB, is an interactive tool for querying, optimizing, and analyzing your MongoDB data. Get key insights, drag and drop to build pipelines, and more. since we are using MongoDB we chose this tool to query and visualize our data. There are other tools such as Studio3T and DataGrip that are used to query and visualize our data but those are commercial and come with a price tag. Compass is freeware and also developed by the team behind MongodDB itself



