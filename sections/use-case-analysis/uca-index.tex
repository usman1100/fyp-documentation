% {
  name: 'Register',
  desc: '',
  actors: '',
  assumptions: '',
  triggers: '',
  pre: '',
  post: '',
  mainCourse: '',
  alternativeCourse: ''
}

    \subsubsection{FR-01: Register}
    \begin{center}
        \begin{tabularx}{\textwidth}{|l|X|}
            \hline
            \textbf{ID} & UC-01 \\
            \hline
            \textbf{Name} & Register \\
            \hline
            \textbf{Description} &  \\
            \hline
            \textbf{Actors} &  \\
            \hline
            \textbf{Assumptions} &  \\
            \hline
            \textbf{Triggers} &  \\
            \hline
            \textbf{Pre Conditions} &  \\
            \hline
            \textbf{Post Conditions} &  \\
            \hline
            \textbf{Main Course} &  \\
            \hline
            \textbf{Alternative Course} &  \\
            \hline
            
        \end{tabularx}
    \end{center}
    
    

    \subsubsection{FR-02: Register}
    \begin{center}
        \begin{tabularx}{\textwidth}{|l|X|}
            \hline
            \textbf{ID} & UC-02 \\
            \hline
            \textbf{Name} & Register \\
            \hline
            \textbf{Description} & User can register their to keep their role in that software \\
            \hline
            \textbf{Actors} & New User \\
            \hline
            \textbf{Assumptions} & If user can Register, he/she will see more features \\
            \hline
            \textbf{Triggers} & Just Connect to in the internet and a email address \\
            \hline
            \textbf{Pre Conditions} & None \\
            \hline
            \textbf{Post Conditions} & Account Created Succesfully \\
            \hline
            \textbf{Main Course} & 1. Enter their Details 2.Create their accound according to their given details \\
            \hline
            \textbf{Alternative Course} & Error due to invalid details \\
            \hline
            
        \end{tabularx}
    \end{center}
    
    

    \subsubsection{FR-03: Login}
    \begin{center}
        \begin{tabularx}{\textwidth}{|l|X|}
            \hline
            \textbf{ID} & UC-03 \\
            \hline
            \textbf{Name} & Login \\
            \hline
            \textbf{Description} & Login to the system to view/change in their account \\
            \hline
            \textbf{Actors} & New User \\
            \hline
            \textbf{Assumptions} & If user can login, he/she will use more features \\
            \hline
            \textbf{Triggers} & just confirm with email which they provided in sign up \\
            \hline
            \textbf{Pre Conditions} & they must have a account \\
            \hline
            \textbf{Post Conditions} & they succesfully see the dashboard and more settings and features \\
            \hline
            \textbf{Main Course} & User can enter/use their email and its valid password Check entered data and able to use \\
            \hline
            \textbf{Alternative Course} & Error due to invalid details \\
            \hline
            
        \end{tabularx}
    \end{center}
    
    

    \subsubsection{FR-04: OAuth}
    \begin{center}
        \begin{tabularx}{\textwidth}{|l|X|}
            \hline
            \textbf{ID} & UC-04 \\
            \hline
            \textbf{Name} & OAuth \\
            \hline
            \textbf{Description} & User can ouath their details with third party like Gmail , LinkdIn and Github \\
            \hline
            \textbf{Actors} & New User \\
            \hline
            \textbf{Assumptions} & User can connect their third party account to login with the software \\
            \hline
            \textbf{Triggers} & just have a third party account  \\
            \hline
            \textbf{Pre Conditions} & none \\
            \hline
            \textbf{Post Conditions} & account has created by confiramtion through email  \\
            \hline
            \textbf{Main Course} & User can create their account on system by sending a email confirmation token. \\
            \hline
            \textbf{Alternative Course} & Error will show if you cant verify your Thir party account \\
            \hline
            
        \end{tabularx}
    \end{center}
    \newpage
    

    \subsubsection{FR-05: Forget Password}
    \begin{center}
        \begin{tabularx}{\textwidth}{|l|X|}
            \hline
            \textbf{ID} & UC-05 \\
            \hline
            \textbf{Name} & Forget Password \\
            \hline
            \textbf{Description} & Send the email to the user with a link to reset their password \\
            \hline
            \textbf{Actors} & Account Holder \\
            \hline
            \textbf{Assumptions} & An email is sent tu the user with a link to reset their password \\
            \hline
            \textbf{Triggers} &  \\
            \hline
            \textbf{Pre Conditions} & Valid Email \\
            \hline
            \textbf{Post Conditions} & New password has been changed \\
            \hline
            \textbf{Main Course} & User can change their password by sending link to their email and setting the new password with the password requirements \\
            \hline
            \textbf{Alternative Course} & Error will be show due to invalid details \\
            \hline
            
        \end{tabularx}
    \end{center}
    
    

    \subsubsection{FR-06: Reset Password}
    \begin{center}
        \begin{tabularx}{\textwidth}{|l|X|}
            \hline
            \textbf{ID} & UC-06 \\
            \hline
            \textbf{Name} & Reset Password \\
            \hline
            \textbf{Description} & User can reset your Password by entering you current password \\
            \hline
            \textbf{Actors} & Account Holder \\
            \hline
            \textbf{Assumptions} & To reset their password, User can enter the curent password and after confirmation it will be default password  \\
            \hline
            \textbf{Triggers} & Account Holder \\
            \hline
            \textbf{Pre Conditions} & user have a account \\
            \hline
            \textbf{Post Conditions} & password has been reset \\
            \hline
            \textbf{Main Course} & User can reset their password by entering their current password and password will set to default Password \\
            \hline
            \textbf{Alternative Course} & Error will be show due to invalid details \\
            \hline
            
        \end{tabularx}
    \end{center}
    
    

    \subsubsection{FR-07: Confirm Account}
    \begin{center}
        \begin{tabularx}{\textwidth}{|l|X|}
            \hline
            \textbf{ID} & UC-07 \\
            \hline
            \textbf{Name} & Confirm Account \\
            \hline
            \textbf{Description} & User can confirm their account by a valid email address \\
            \hline
            \textbf{Actors} & New User \\
            \hline
            \textbf{Assumptions} &  \\
            \hline
            \textbf{Triggers} & By send the valid link to user's provided email \\
            \hline
            \textbf{Pre Conditions} & Fill the signup form \\
            \hline
            \textbf{Post Conditions} & account verified \\
            \hline
            \textbf{Main Course} & Account has been created though 3rd Party validation \\
            \hline
            \textbf{Alternative Course} & Error by unvalid email provided \\
            \hline
            
        \end{tabularx}
    \end{center}
    
    

    \subsubsection{FR-08: Delete Account}
    \begin{center}
        \begin{tabularx}{\textwidth}{|l|X|}
            \hline
            \textbf{ID} & UC-08 \\
            \hline
            \textbf{Name} & Delete Account \\
            \hline
            \textbf{Description} & User can delete the account if he/she can \\
            \hline
            \textbf{Actors} & Already User \\
            \hline
            \textbf{Assumptions} &  \\
            \hline
            \textbf{Triggers} & remove the user\_id from database by click on reset button \\
            \hline
            \textbf{Pre Conditions} & Already user database was be present in database \\
            \hline
            \textbf{Post Conditions} & No user of that user\_id will not be in database \\
            \hline
            \textbf{Main Course} & Account has been removed from database by the confirmation of user \\
            \hline
            \textbf{Alternative Course} & Error will be appeared by database \\
            \hline
            
        \end{tabularx}
    \end{center}
    \newpage
    

    \subsubsection{FR-09: Edit Account Info}
    \begin{center}
        \begin{tabularx}{\textwidth}{|l|X|}
            \hline
            \textbf{ID} & UC-09 \\
            \hline
            \textbf{Name} & Edit Account Info \\
            \hline
            \textbf{Description} & User can edit their account information \\
            \hline
            \textbf{Actors} & Already User \\
            \hline
            \textbf{Assumptions} &  \\
            \hline
            \textbf{Triggers} & Database of that user can be updated \\
            \hline
            \textbf{Pre Conditions} & Already data of that user will be in database  \\
            \hline
            \textbf{Post Conditions} & Update the new version of their information in database \\
            \hline
            \textbf{Main Course} & Information of the User can be updated if he/she can \\
            \hline
            \textbf{Alternative Course} & Error willbe displaced by the Database \\
            \hline
            
        \end{tabularx}
    \end{center}
    
    

    \subsubsection{FR-10: Edit Interface Preferences}
    \begin{center}
        \begin{tabularx}{\textwidth}{|l|X|}
            \hline
            \textbf{ID} & UC-10 \\
            \hline
            \textbf{Name} & Edit Interface Preferences \\
            \hline
            \textbf{Description} & User can edit their preferences of the UI of their dashboard \\
            \hline
            \textbf{Actors} & User \\
            \hline
            \textbf{Assumptions} &  \\
            \hline
            \textbf{Triggers} & User can change the primary and secondary Colors of the themes \\
            \hline
            \textbf{Pre Conditions} & Valid User \\
            \hline
            \textbf{Post Conditions} & UI preferences will be change according to User \\
            \hline
            \textbf{Main Course} & Interface preference will be changed by the user according to their preception  \\
            \hline
            \textbf{Alternative Course} & Error will be displayed it choose certain colors like black etc. \\
            \hline
            
        \end{tabularx}
    \end{center}
    
    

    \subsubsection{FR-11: Create Project}
    \begin{center}
        \begin{tabularx}{\textwidth}{|l|X|}
            \hline
            \textbf{ID} & UC-11 \\
            \hline
            \textbf{Name} & Create Project \\
            \hline
            \textbf{Description} & Manager can create a project which is stored in the database with the unique id \\
            \hline
            \textbf{Actors} & Manager \\
            \hline
            \textbf{Assumptions} &  \\
            \hline
            \textbf{Triggers} & The Project will created in the database with unique id \\
            \hline
            \textbf{Pre Conditions} & user will be valid and Project can't created before  \\
            \hline
            \textbf{Post Conditions} & Project is created by the Manager \\
            \hline
            \textbf{Main Course} & Project is created by the manager in database \\
            \hline
            \textbf{Alternative Course} & Error will be displayed of database  \\
            \hline
            
        \end{tabularx}
    \end{center}
    
    

    \subsubsection{FR-12: Edit Project}
    \begin{center}
        \begin{tabularx}{\textwidth}{|l|X|}
            \hline
            \textbf{ID} & UC-12 \\
            \hline
            \textbf{Name} & Edit Project \\
            \hline
            \textbf{Description} & The existing project will be edited by the manager if needed  \\
            \hline
            \textbf{Actors} & Manager \\
            \hline
            \textbf{Assumptions} &  \\
            \hline
            \textbf{Triggers} & The Information of the existing project will be updated in the database \\
            \hline
            \textbf{Pre Conditions} & Project is present already in database \\
            \hline
            \textbf{Post Conditions} & Project's Information will be updated by manager \\
            \hline
            \textbf{Main Course} & Manager can change the existing project if the requirements will be changed by the client \\
            \hline
            \textbf{Alternative Course} & Database's error will occurs \\
            \hline
            
        \end{tabularx}
    \end{center}
    \newpage
    

    \subsubsection{FR-13: Delete Project}
    \begin{center}
        \begin{tabularx}{\textwidth}{|l|X|}
            \hline
            \textbf{ID} & UC-13 \\
            \hline
            \textbf{Name} & Delete Project \\
            \hline
            \textbf{Description} & The Manager will delete the existing projects from the database \\
            \hline
            \textbf{Actors} & Managers \\
            \hline
            \textbf{Assumptions} &  \\
            \hline
            \textbf{Triggers} & Delete the existing project from the database by delete event \\
            \hline
            \textbf{Pre Conditions} & Project will be in the database which Manager will be deleted \\
            \hline
            \textbf{Post Conditions} & Project is deletedby the the manager \\
            \hline
            \textbf{Main Course} & The manager will delete the exiting project which is presnet in database \\
            \hline
            \textbf{Alternative Course} & Error will be displayed \\
            \hline
            
        \end{tabularx}
    \end{center}
    
    

    \subsubsection{FR-14: Add Developer}
    \begin{center}
        \begin{tabularx}{\textwidth}{|l|X|}
            \hline
            \textbf{ID} & UC-14 \\
            \hline
            \textbf{Name} & Add Developer \\
            \hline
            \textbf{Description} & The Manager will assign the developer of the company which is part of that Project \\
            \hline
            \textbf{Actors} & Managers \\
            \hline
            \textbf{Assumptions} &  \\
            \hline
            \textbf{Triggers} & Developer list will be shown to manager and manager will add the relevant developer \\
            \hline
            \textbf{Pre Conditions} & Developer and project will be added \\
            \hline
            \textbf{Post Conditions} & Developer will be added in the specific project \\
            \hline
            \textbf{Main Course} & To view the new upcoming project and take the decision of developer in project, Manager will added the developer in the project \\
            \hline
            \textbf{Alternative Course} & Error will be displayed \\
            \hline
            
        \end{tabularx}
    \end{center}
    
    

    \subsubsection{FR-15: Remove Developer}
    \begin{center}
        \begin{tabularx}{\textwidth}{|l|X|}
            \hline
            \textbf{ID} & UC-15 \\
            \hline
            \textbf{Name} & Remove Developer \\
            \hline
            \textbf{Description} & The manager can remove the developer after developer performs their duties \\
            \hline
            \textbf{Actors} & Manager \\
            \hline
            \textbf{Assumptions} &  \\
            \hline
            \textbf{Triggers} & Manager will remove the developer in the project anytime. \\
            \hline
            \textbf{Pre Conditions} & Developer will be added in that project \\
            \hline
            \textbf{Post Conditions} & developer is no more the part of the project \\
            \hline
            \textbf{Main Course} & Manager can manage the availabilty of the developer whenever he enter in the project or when he exits. \\
            \hline
            \textbf{Alternative Course} & Error will be displayed \\
            \hline
            
        \end{tabularx}
    \end{center}
    
    

    \subsubsection{FR-16: Edit Developer Permission}
    \begin{center}
        \begin{tabularx}{\textwidth}{|l|X|}
            \hline
            \textbf{ID} & UC-16 \\
            \hline
            \textbf{Name} & Edit Developer Permission \\
            \hline
            \textbf{Description} & The Manager can edit the role of the developer in the project \\
            \hline
            \textbf{Actors} & Managers \\
            \hline
            \textbf{Assumptions} &  \\
            \hline
            \textbf{Triggers} & Manager can edit the preference of the developer \\
            \hline
            \textbf{Pre Conditions} & Developer must have the part of the project \\
            \hline
            \textbf{Post Conditions} & Developer's Role will changed by the manager \\
            \hline
            \textbf{Main Course} & The Developer can change the role of the developer as he needs him in the project \\
            \hline
            \textbf{Alternative Course} & Error will be dislpayed by the database \\
            \hline
            
        \end{tabularx}
    \end{center}
    \newpage
    

    \subsubsection{FR-17: Calculate UCP}
    \begin{center}
        \begin{tabularx}{\textwidth}{|l|X|}
            \hline
            \textbf{ID} & UC-17 \\
            \hline
            \textbf{Name} & Calculate UCP \\
            \hline
            \textbf{Description} & Calculate the effort estimation of project via mean of UCP method \\
            \hline
            \textbf{Actors} & Manager , Developers \\
            \hline
            \textbf{Assumptions} &  \\
            \hline
            \textbf{Triggers} & The effort will calculted by UCP formula \\
            \hline
            \textbf{Pre Conditions} & The Project Use cases' information are inserted already \\
            \hline
            \textbf{Post Conditions} & Calculated Effort are given by UCP method  \\
            \hline
            \textbf{Main Course} & To find the Effort Estimation by UCP method of the entire project by the help of their use cases \\
            \hline
            \textbf{Alternative Course} & Error will displayed \\
            \hline
            
        \end{tabularx}
    \end{center}
    
    

    \subsubsection{FR-18: Get Machine Learning Estimation}
    \begin{center}
        \begin{tabularx}{\textwidth}{|l|X|}
            \hline
            \textbf{ID} & UC-18 \\
            \hline
            \textbf{Name} & Get Machine Learning Estimation \\
            \hline
            \textbf{Description} & Calculate the effort estimation of project via trained Machine Learning module \\
            \hline
            \textbf{Actors} & Manager  \\
            \hline
            \textbf{Assumptions} &  \\
            \hline
            \textbf{Triggers} & The effort will calculted by UCP formula \\
            \hline
            \textbf{Pre Conditions} & The Project Use cases' information are inserted already \\
            \hline
            \textbf{Post Conditions} & Calculated Effort are given by UCP method  \\
            \hline
            \textbf{Main Course} & To find the Effort Estimation by UCP method of the entire project by the help of their use cases \\
            \hline
            \textbf{Alternative Course} & Error will displayed \\
            \hline
            
        \end{tabularx}
    \end{center}
    
    

    \subsubsection{FR-19: Manual Estimate Round}
    \begin{center}
        \begin{tabularx}{\textwidth}{|l|X|}
            \hline
            \textbf{ID} & UC-19 \\
            \hline
            \textbf{Name} & Manual Estimate Round \\
            \hline
            \textbf{Description} & Calculate the effort estimation of project via Manual Estimation where the developer and experts will estimate the project while round \\
            \hline
            \textbf{Actors} & developers, manager \\
            \hline
            \textbf{Assumptions} &  \\
            \hline
            \textbf{Triggers} & The Effort will calculated by Manual technique \\
            \hline
            \textbf{Pre Conditions} & Developers were added by manager  in the project \\
            \hline
            \textbf{Post Conditions} & The effort will calculated after the many rounds of the experts' discussions  \\
            \hline
            \textbf{Main Course} & To find the Effort estimation by Manual technique by many rounds \\
            \hline
            \textbf{Alternative Course} & Error will displayed \\
            \hline
            
        \end{tabularx}
    \end{center}
    
    

    \subsubsection{FR-20: Estimation Bar}
    \begin{center}
        \begin{tabularx}{\textwidth}{|l|X|}
            \hline
            \textbf{ID} & UC-20 \\
            \hline
            \textbf{Name} & Estimation Bar \\
            \hline
            \textbf{Description} & All Estimation will be shown in the form of graphs \\
            \hline
            \textbf{Actors} & Manager , Developer \\
            \hline
            \textbf{Assumptions} &  \\
            \hline
            \textbf{Triggers} & The Estimate will be generated by the experts' round discussion \\
            \hline
            \textbf{Pre Conditions} & The estimation will be measured before that step  \\
            \hline
            \textbf{Post Conditions} & The charts will be appeared \\
            \hline
            \textbf{Main Course} & Estimation will be deliver in bar that anyone will be see and clearly judge the estimation of the project \\
            \hline
            \textbf{Alternative Course} & Error wil be displayed \\
            \hline
            
        \end{tabularx}
    \end{center}
    \newpage
    


\section{Use Case Analysis}

\vspace{20mm}

\begin{abstract}
The needs of a system, including internal and external factors, 
are gathered via use case diagrams. The majority of these requirements are 
design-related. As a result, use cases are generated and actors are identified 
when a system is studied to gather its functionality. The purpose of this study
is to determine the value of use case diagrams in software development. 
According to the findings of this article, the importance of the use case
diagram grows in proportion to the size of the project. Because the size 
of a project can be directly tied to its level of complexity. Large and 
complex projects always require the usage of a use case diagram so that 
engineers can quickly grasp the system's requirements.
A use case analysis is the primary form for
gathering usage requirements for a new
software program or task to be completed.
\vspace{2mm}

\textbf{Primary Goals:}

\begin{center}
    \begin{itemize}
        \item  designing a system from the user's perspective
        \item communicating system behavior in the user's terms
        \item specifying all externally visible behaviors
    \end{itemize}
\end{center}

    
\end{abstract}


\vspace{20mm}

\large{\textbf{Outline}}

\begin{center}
    \begin{itemize}
        \item Manager Use Cases
        \item Developer Use Cases
        \item Miscellaneous Use Cases
    \end{itemize}
\end{center}

\pagebreak

\subsection{Use Case Analysis}
{
  name: 'Register',
  desc: '',
  actors: '',
  assumptions: '',
  triggers: '',
  pre: '',
  post: '',
  mainCourse: '',
  alternativeCourse: ''
}

    \subsubsection{FR-01: Register}
    \begin{center}
        \begin{tabularx}{\textwidth}{|l|X|}
            \hline
            \textbf{ID} & UC-01 \\
            \hline
            \textbf{Name} & Register \\
            \hline
            \textbf{Description} &  \\
            \hline
            \textbf{Actors} &  \\
            \hline
            \textbf{Assumptions} &  \\
            \hline
            \textbf{Triggers} &  \\
            \hline
            \textbf{Pre Conditions} &  \\
            \hline
            \textbf{Post Conditions} &  \\
            \hline
            \textbf{Main Course} &  \\
            \hline
            \textbf{Alternative Course} &  \\
            \hline
            
        \end{tabularx}
    \end{center}
    
    

    \subsubsection{FR-02: Register}
    \begin{center}
        \begin{tabularx}{\textwidth}{|l|X|}
            \hline
            \textbf{ID} & UC-02 \\
            \hline
            \textbf{Name} & Register \\
            \hline
            \textbf{Description} & User can register their to keep their role in that software \\
            \hline
            \textbf{Actors} & New User \\
            \hline
            \textbf{Assumptions} & If user can Register, he/she will see more features \\
            \hline
            \textbf{Triggers} & Just Connect to in the internet and a email address \\
            \hline
            \textbf{Pre Conditions} & None \\
            \hline
            \textbf{Post Conditions} & Account Created Succesfully \\
            \hline
            \textbf{Main Course} & 1. Enter their Details 2.Create their accound according to their given details \\
            \hline
            \textbf{Alternative Course} & Error due to invalid details \\
            \hline
            
        \end{tabularx}
    \end{center}
    
    

    \subsubsection{FR-03: Login}
    \begin{center}
        \begin{tabularx}{\textwidth}{|l|X|}
            \hline
            \textbf{ID} & UC-03 \\
            \hline
            \textbf{Name} & Login \\
            \hline
            \textbf{Description} & Login to the system to view/change in their account \\
            \hline
            \textbf{Actors} & New User \\
            \hline
            \textbf{Assumptions} & If user can login, he/she will use more features \\
            \hline
            \textbf{Triggers} & just confirm with email which they provided in sign up \\
            \hline
            \textbf{Pre Conditions} & they must have a account \\
            \hline
            \textbf{Post Conditions} & they succesfully see the dashboard and more settings and features \\
            \hline
            \textbf{Main Course} & User can enter/use their email and its valid password Check entered data and able to use \\
            \hline
            \textbf{Alternative Course} & Error due to invalid details \\
            \hline
            
        \end{tabularx}
    \end{center}
    
    

    \subsubsection{FR-04: OAuth}
    \begin{center}
        \begin{tabularx}{\textwidth}{|l|X|}
            \hline
            \textbf{ID} & UC-04 \\
            \hline
            \textbf{Name} & OAuth \\
            \hline
            \textbf{Description} & User can ouath their details with third party like Gmail , LinkdIn and Github \\
            \hline
            \textbf{Actors} & New User \\
            \hline
            \textbf{Assumptions} & User can connect their third party account to login with the software \\
            \hline
            \textbf{Triggers} & just have a third party account  \\
            \hline
            \textbf{Pre Conditions} & none \\
            \hline
            \textbf{Post Conditions} & account has created by confiramtion through email  \\
            \hline
            \textbf{Main Course} & User can create their account on system by sending a email confirmation token. \\
            \hline
            \textbf{Alternative Course} & Error will show if you cant verify your Thir party account \\
            \hline
            
        \end{tabularx}
    \end{center}
    \newpage
    

    \subsubsection{FR-05: Forget Password}
    \begin{center}
        \begin{tabularx}{\textwidth}{|l|X|}
            \hline
            \textbf{ID} & UC-05 \\
            \hline
            \textbf{Name} & Forget Password \\
            \hline
            \textbf{Description} & Send the email to the user with a link to reset their password \\
            \hline
            \textbf{Actors} & Account Holder \\
            \hline
            \textbf{Assumptions} & An email is sent tu the user with a link to reset their password \\
            \hline
            \textbf{Triggers} &  \\
            \hline
            \textbf{Pre Conditions} & Valid Email \\
            \hline
            \textbf{Post Conditions} & New password has been changed \\
            \hline
            \textbf{Main Course} & User can change their password by sending link to their email and setting the new password with the password requirements \\
            \hline
            \textbf{Alternative Course} & Error will be show due to invalid details \\
            \hline
            
        \end{tabularx}
    \end{center}
    
    

    \subsubsection{FR-06: Reset Password}
    \begin{center}
        \begin{tabularx}{\textwidth}{|l|X|}
            \hline
            \textbf{ID} & UC-06 \\
            \hline
            \textbf{Name} & Reset Password \\
            \hline
            \textbf{Description} & User can reset your Password by entering you current password \\
            \hline
            \textbf{Actors} & Account Holder \\
            \hline
            \textbf{Assumptions} & To reset their password, User can enter the curent password and after confirmation it will be default password  \\
            \hline
            \textbf{Triggers} & Account Holder \\
            \hline
            \textbf{Pre Conditions} & user have a account \\
            \hline
            \textbf{Post Conditions} & password has been reset \\
            \hline
            \textbf{Main Course} & User can reset their password by entering their current password and password will set to default Password \\
            \hline
            \textbf{Alternative Course} & Error will be show due to invalid details \\
            \hline
            
        \end{tabularx}
    \end{center}
    
    

    \subsubsection{FR-07: Confirm Account}
    \begin{center}
        \begin{tabularx}{\textwidth}{|l|X|}
            \hline
            \textbf{ID} & UC-07 \\
            \hline
            \textbf{Name} & Confirm Account \\
            \hline
            \textbf{Description} & User can confirm their account by a valid email address \\
            \hline
            \textbf{Actors} & New User \\
            \hline
            \textbf{Assumptions} &  \\
            \hline
            \textbf{Triggers} & By send the valid link to user's provided email \\
            \hline
            \textbf{Pre Conditions} & Fill the signup form \\
            \hline
            \textbf{Post Conditions} & account verified \\
            \hline
            \textbf{Main Course} & Account has been created though 3rd Party validation \\
            \hline
            \textbf{Alternative Course} & Error by unvalid email provided \\
            \hline
            
        \end{tabularx}
    \end{center}
    
    

    \subsubsection{FR-08: Delete Account}
    \begin{center}
        \begin{tabularx}{\textwidth}{|l|X|}
            \hline
            \textbf{ID} & UC-08 \\
            \hline
            \textbf{Name} & Delete Account \\
            \hline
            \textbf{Description} & User can delete the account if he/she can \\
            \hline
            \textbf{Actors} & Already User \\
            \hline
            \textbf{Assumptions} &  \\
            \hline
            \textbf{Triggers} & remove the user\_id from database by click on reset button \\
            \hline
            \textbf{Pre Conditions} & Already user database was be present in database \\
            \hline
            \textbf{Post Conditions} & No user of that user\_id will not be in database \\
            \hline
            \textbf{Main Course} & Account has been removed from database by the confirmation of user \\
            \hline
            \textbf{Alternative Course} & Error will be appeared by database \\
            \hline
            
        \end{tabularx}
    \end{center}
    \newpage
    

    \subsubsection{FR-09: Edit Account Info}
    \begin{center}
        \begin{tabularx}{\textwidth}{|l|X|}
            \hline
            \textbf{ID} & UC-09 \\
            \hline
            \textbf{Name} & Edit Account Info \\
            \hline
            \textbf{Description} & User can edit their account information \\
            \hline
            \textbf{Actors} & Already User \\
            \hline
            \textbf{Assumptions} &  \\
            \hline
            \textbf{Triggers} & Database of that user can be updated \\
            \hline
            \textbf{Pre Conditions} & Already data of that user will be in database  \\
            \hline
            \textbf{Post Conditions} & Update the new version of their information in database \\
            \hline
            \textbf{Main Course} & Information of the User can be updated if he/she can \\
            \hline
            \textbf{Alternative Course} & Error willbe displaced by the Database \\
            \hline
            
        \end{tabularx}
    \end{center}
    
    

    \subsubsection{FR-10: Edit Interface Preferences}
    \begin{center}
        \begin{tabularx}{\textwidth}{|l|X|}
            \hline
            \textbf{ID} & UC-10 \\
            \hline
            \textbf{Name} & Edit Interface Preferences \\
            \hline
            \textbf{Description} & User can edit their preferences of the UI of their dashboard \\
            \hline
            \textbf{Actors} & User \\
            \hline
            \textbf{Assumptions} &  \\
            \hline
            \textbf{Triggers} & User can change the primary and secondary Colors of the themes \\
            \hline
            \textbf{Pre Conditions} & Valid User \\
            \hline
            \textbf{Post Conditions} & UI preferences will be change according to User \\
            \hline
            \textbf{Main Course} & Interface preference will be changed by the user according to their preception  \\
            \hline
            \textbf{Alternative Course} & Error will be displayed it choose certain colors like black etc. \\
            \hline
            
        \end{tabularx}
    \end{center}
    
    

    \subsubsection{FR-11: Create Project}
    \begin{center}
        \begin{tabularx}{\textwidth}{|l|X|}
            \hline
            \textbf{ID} & UC-11 \\
            \hline
            \textbf{Name} & Create Project \\
            \hline
            \textbf{Description} & Manager can create a project which is stored in the database with the unique id \\
            \hline
            \textbf{Actors} & Manager \\
            \hline
            \textbf{Assumptions} &  \\
            \hline
            \textbf{Triggers} & The Project will created in the database with unique id \\
            \hline
            \textbf{Pre Conditions} & user will be valid and Project can't created before  \\
            \hline
            \textbf{Post Conditions} & Project is created by the Manager \\
            \hline
            \textbf{Main Course} & Project is created by the manager in database \\
            \hline
            \textbf{Alternative Course} & Error will be displayed of database  \\
            \hline
            
        \end{tabularx}
    \end{center}
    
    

    \subsubsection{FR-12: Edit Project}
    \begin{center}
        \begin{tabularx}{\textwidth}{|l|X|}
            \hline
            \textbf{ID} & UC-12 \\
            \hline
            \textbf{Name} & Edit Project \\
            \hline
            \textbf{Description} & The existing project will be edited by the manager if needed  \\
            \hline
            \textbf{Actors} & Manager \\
            \hline
            \textbf{Assumptions} &  \\
            \hline
            \textbf{Triggers} & The Information of the existing project will be updated in the database \\
            \hline
            \textbf{Pre Conditions} & Project is present already in database \\
            \hline
            \textbf{Post Conditions} & Project's Information will be updated by manager \\
            \hline
            \textbf{Main Course} & Manager can change the existing project if the requirements will be changed by the client \\
            \hline
            \textbf{Alternative Course} & Database's error will occurs \\
            \hline
            
        \end{tabularx}
    \end{center}
    \newpage
    

    \subsubsection{FR-13: Delete Project}
    \begin{center}
        \begin{tabularx}{\textwidth}{|l|X|}
            \hline
            \textbf{ID} & UC-13 \\
            \hline
            \textbf{Name} & Delete Project \\
            \hline
            \textbf{Description} & The Manager will delete the existing projects from the database \\
            \hline
            \textbf{Actors} & Managers \\
            \hline
            \textbf{Assumptions} &  \\
            \hline
            \textbf{Triggers} & Delete the existing project from the database by delete event \\
            \hline
            \textbf{Pre Conditions} & Project will be in the database which Manager will be deleted \\
            \hline
            \textbf{Post Conditions} & Project is deletedby the the manager \\
            \hline
            \textbf{Main Course} & The manager will delete the exiting project which is presnet in database \\
            \hline
            \textbf{Alternative Course} & Error will be displayed \\
            \hline
            
        \end{tabularx}
    \end{center}
    
    

    \subsubsection{FR-14: Add Developer}
    \begin{center}
        \begin{tabularx}{\textwidth}{|l|X|}
            \hline
            \textbf{ID} & UC-14 \\
            \hline
            \textbf{Name} & Add Developer \\
            \hline
            \textbf{Description} & The Manager will assign the developer of the company which is part of that Project \\
            \hline
            \textbf{Actors} & Managers \\
            \hline
            \textbf{Assumptions} &  \\
            \hline
            \textbf{Triggers} & Developer list will be shown to manager and manager will add the relevant developer \\
            \hline
            \textbf{Pre Conditions} & Developer and project will be added \\
            \hline
            \textbf{Post Conditions} & Developer will be added in the specific project \\
            \hline
            \textbf{Main Course} & To view the new upcoming project and take the decision of developer in project, Manager will added the developer in the project \\
            \hline
            \textbf{Alternative Course} & Error will be displayed \\
            \hline
            
        \end{tabularx}
    \end{center}
    
    

    \subsubsection{FR-15: Remove Developer}
    \begin{center}
        \begin{tabularx}{\textwidth}{|l|X|}
            \hline
            \textbf{ID} & UC-15 \\
            \hline
            \textbf{Name} & Remove Developer \\
            \hline
            \textbf{Description} & The manager can remove the developer after developer performs their duties \\
            \hline
            \textbf{Actors} & Manager \\
            \hline
            \textbf{Assumptions} &  \\
            \hline
            \textbf{Triggers} & Manager will remove the developer in the project anytime. \\
            \hline
            \textbf{Pre Conditions} & Developer will be added in that project \\
            \hline
            \textbf{Post Conditions} & developer is no more the part of the project \\
            \hline
            \textbf{Main Course} & Manager can manage the availabilty of the developer whenever he enter in the project or when he exits. \\
            \hline
            \textbf{Alternative Course} & Error will be displayed \\
            \hline
            
        \end{tabularx}
    \end{center}
    
    

    \subsubsection{FR-16: Edit Developer Permission}
    \begin{center}
        \begin{tabularx}{\textwidth}{|l|X|}
            \hline
            \textbf{ID} & UC-16 \\
            \hline
            \textbf{Name} & Edit Developer Permission \\
            \hline
            \textbf{Description} & The Manager can edit the role of the developer in the project \\
            \hline
            \textbf{Actors} & Managers \\
            \hline
            \textbf{Assumptions} &  \\
            \hline
            \textbf{Triggers} & Manager can edit the preference of the developer \\
            \hline
            \textbf{Pre Conditions} & Developer must have the part of the project \\
            \hline
            \textbf{Post Conditions} & Developer's Role will changed by the manager \\
            \hline
            \textbf{Main Course} & The Developer can change the role of the developer as he needs him in the project \\
            \hline
            \textbf{Alternative Course} & Error will be dislpayed by the database \\
            \hline
            
        \end{tabularx}
    \end{center}
    \newpage
    

    \subsubsection{FR-17: Calculate UCP}
    \begin{center}
        \begin{tabularx}{\textwidth}{|l|X|}
            \hline
            \textbf{ID} & UC-17 \\
            \hline
            \textbf{Name} & Calculate UCP \\
            \hline
            \textbf{Description} & Calculate the effort estimation of project via mean of UCP method \\
            \hline
            \textbf{Actors} & Manager , Developers \\
            \hline
            \textbf{Assumptions} &  \\
            \hline
            \textbf{Triggers} & The effort will calculted by UCP formula \\
            \hline
            \textbf{Pre Conditions} & The Project Use cases' information are inserted already \\
            \hline
            \textbf{Post Conditions} & Calculated Effort are given by UCP method  \\
            \hline
            \textbf{Main Course} & To find the Effort Estimation by UCP method of the entire project by the help of their use cases \\
            \hline
            \textbf{Alternative Course} & Error will displayed \\
            \hline
            
        \end{tabularx}
    \end{center}
    
    

    \subsubsection{FR-18: Get Machine Learning Estimation}
    \begin{center}
        \begin{tabularx}{\textwidth}{|l|X|}
            \hline
            \textbf{ID} & UC-18 \\
            \hline
            \textbf{Name} & Get Machine Learning Estimation \\
            \hline
            \textbf{Description} & Calculate the effort estimation of project via trained Machine Learning module \\
            \hline
            \textbf{Actors} & Manager  \\
            \hline
            \textbf{Assumptions} &  \\
            \hline
            \textbf{Triggers} & The effort will calculted by UCP formula \\
            \hline
            \textbf{Pre Conditions} & The Project Use cases' information are inserted already \\
            \hline
            \textbf{Post Conditions} & Calculated Effort are given by UCP method  \\
            \hline
            \textbf{Main Course} & To find the Effort Estimation by UCP method of the entire project by the help of their use cases \\
            \hline
            \textbf{Alternative Course} & Error will displayed \\
            \hline
            
        \end{tabularx}
    \end{center}
    
    

    \subsubsection{FR-19: Manual Estimate Round}
    \begin{center}
        \begin{tabularx}{\textwidth}{|l|X|}
            \hline
            \textbf{ID} & UC-19 \\
            \hline
            \textbf{Name} & Manual Estimate Round \\
            \hline
            \textbf{Description} & Calculate the effort estimation of project via Manual Estimation where the developer and experts will estimate the project while round \\
            \hline
            \textbf{Actors} & developers, manager \\
            \hline
            \textbf{Assumptions} &  \\
            \hline
            \textbf{Triggers} & The Effort will calculated by Manual technique \\
            \hline
            \textbf{Pre Conditions} & Developers were added by manager  in the project \\
            \hline
            \textbf{Post Conditions} & The effort will calculated after the many rounds of the experts' discussions  \\
            \hline
            \textbf{Main Course} & To find the Effort estimation by Manual technique by many rounds \\
            \hline
            \textbf{Alternative Course} & Error will displayed \\
            \hline
            
        \end{tabularx}
    \end{center}
    
    

    \subsubsection{FR-20: Estimation Bar}
    \begin{center}
        \begin{tabularx}{\textwidth}{|l|X|}
            \hline
            \textbf{ID} & UC-20 \\
            \hline
            \textbf{Name} & Estimation Bar \\
            \hline
            \textbf{Description} & All Estimation will be shown in the form of graphs \\
            \hline
            \textbf{Actors} & Manager , Developer \\
            \hline
            \textbf{Assumptions} &  \\
            \hline
            \textbf{Triggers} & The Estimate will be generated by the experts' round discussion \\
            \hline
            \textbf{Pre Conditions} & The estimation will be measured before that step  \\
            \hline
            \textbf{Post Conditions} & The charts will be appeared \\
            \hline
            \textbf{Main Course} & Estimation will be deliver in bar that anyone will be see and clearly judge the estimation of the project \\
            \hline
            \textbf{Alternative Course} & Error wil be displayed \\
            \hline
            
        \end{tabularx}
    \end{center}
    \newpage
    


\subsection{Agregated Use Case Diagram}
\begin{figure}[H]
    \includegraphics[height=23cm, width=0.8\textwidth]{./diagrams/Use Case/agrregated new.png}
    \centering 
    \caption{Agregated Use Case Diagram}
    \label{figurea}
    \end{figure}
    