\section{Introduction}

\vspace{20mm}

\Huge{\textbf{Introduction to Problem}}

\vspace{20mm}


\begin{abstract}
	In this chapter, we will be introducing the problem and the requirements that will be used to solve it. The purpose and main objectives that are at the core of the project will be explained in a concise manner. Along with those details we will also discuss the already existing solutions to the current problem and how these existing solutions are no longer a viable choice for consumers. We shall also discuss how our solution fixes the issues that were found in the existing solutions and how our solution/product will be a superior and generally better. After these details an executive summary will summarize all of the above discussions into a concise manner.
\end{abstract}

\vspace{20mm}






\large{\textbf{Outline}}

\begin{center}
	\begin{itemize}
		\item Introduction
		\item Purpose
		\item Objectives
		\item Existing Solutions
		\item Proposed Solution
		\item Novelty
		\item Executive Summary
	\end{itemize}
\end{center}
\pagebreak








\subsection{Introduction}
The software industry is a multi billion dollar industry, in-fact it is projected to be around 1 trillion USD in 2023. This is due to the fact that software development is complex and requires a lot of resources. And just like any complex engineering project, much of these resources, time and effort goes into planning the development of the project. Successfully software developers and companies realize the importance of a well organized and well planned software development process. Without one, the project will more than likely fail to be completed. {\it{"Without requirements or design , programming is the art of adding bugs to an empty text file"}}, a humorous quote from Louis Srygley, well summarizes the need for  a well planned and estimated software development process.

After all the planning and design phases, team leads and managers have to come up with a time and resource estimate for the project. This is a crucial phase and a lot can go wrong here. According to a survey over 60\% of software projects fail to complete in the estimated time and budget. 












\subsection{Purpose}
The main purpose of our product is to improve the accuracy of estimate predictions. This will be done using state-of-the-art machine learning algorithms that are trained on previous software development projects. The classifier will also improve significantly with the passage of time, since we will be recording statistics of all the projects that will be entered in the product. 

 The secondary purpose is to provide a platform where developers and managers can collaborate and share their options on the project's development and create estimates manually. Our product will basically give it's users the option to choose between different estimation techniques but the recommended technique will still be the machine learning classifier.s












\subsection{Objective}
Following are the most important objectives of the project.
\begin{itemize}
	\item Improve the accuracy of software cost and effort estimation techniques.
	\item Get up-to 50\% of software development organizations to switch to automated software cost and effort estimation techniques.
	\item Get managers to switch to our product by automating their current estimation workflows and then  integrating our specialized techniques with their existing workflows.
	\item Making our product compatible with most of today's trending programming languages and their popular frameworks.
	\item Making our product relevant among the software development organizations and get 20 large software development organizations and/or 50 medium scale organizations to switch to our product by the end of 2023. And to integrate our product with their existing management routines.
\end{itemize}













\subsection{Existing Solutions}
There are a few existing solutions in this problem space. Some of them are now not maintained and are deprecated. Following is a list of similar projects
\begin{itemize}
	\item {\bf{SEER For Software}}\newline
	SEER is a general purpose estimation tool that also has a product for software projects' estimation. It accepts input in form of SLOC (Source lines of code), function points, use cases and some more less popular options. After processing these inputs using proprietary models, the output of the program is as follows: Project time duration, development hours and accuracy of estimations.

	\item {\bf{TruePlanning \small{\textregistered}}}\newline
	This product is calculates its estimates using the PRICE model. Developed in 1975, TruePlanning is also a general purpose estimation system.
	
\end{itemize}













\subsection{Disadvantages}
Following are some serious shortcomings or disadvantages with the currently existing solutions
\begin{itemize}
	\item Outdated models that cannot adapt to today's rapidly changing and extremely diverse development languages and frameworks.
	\item No platform to enable communication between management and developers to discuss and agree to a better estimate
	\item Older and less commonly used input formats like SLOC and function points
	\item Outdated subscription methods. All of these programs implement a {\it{One Time Payment}} method. The disadvantage here is that the one time payment is a huge sum of money and may throw off a potential customer's interest in the product.
\end{itemize}









\subsection{Proposed Solution}
The proposed solution is a web-based SPA. This application will be a platform for performing all sorts of estimations for your software project. There will be two types of users, managers and developers. A manager can create a "project" and assign developers into it. The assigned developers can then give their estimates in different "rounds". After the estimates of all developers are close enough then it would regarded as the agreed estimate.

How ever if the team chooses to go towards the automatic technique, they will have to enter their project's attributes in form of "use case points". After that they will enter the "stack" that they are using to develop their project which will help set the environmental factors and productivity factors. The system will then calculate the estimates for the project using either the simple UCP technique or the machine learning methods (according to the user's preferences). The estimates will be displayed in a various forms like tables, graphs and charts.







\subsection{Novelty}
Here are some points that make our product a better solution than the existing ones mentioned above.

\begin{itemize}
	\item {\bfseries Latest data:} Our predictions and estimates are based on the data that is collected from the past few years. This data is obtained from the UCP dataset. Along with that, we have also expanded the UCP dataset from 72 to 99 projects by collecting data from some development organizations. Newer projects are created using latest technologies and frameworks hence they are a better fit to compare with the future projects. Also the product itself will be collecting data from all of it's customer for the sake of self improvement.
	
	
	\item {\bfseries Machine Learning:} In the past all manual or automated estimations were done using simple mathematical formulas and techniques. This approach is very rigid as it doesn't leave much room for flexibility. With machine learning techniques, no two estimations are the same. The system will be able to learn from the past data and make better estimates.
	

	\item {\bfseries Platform to collaborate:} All of the cost estimation services will be provided on a platform that is built for the collaboration of developers and managers. Here you can estimate your project and along with that, get statistics of your project. You can manage all your project at one place. Generate graphs and tables of your project's estimates to extract helpful insights from it.
	

	\item {\bfseries Multiple Methods:} Initially we will be providing three different techniques for estimations. First is the "Delphi Technique" for manual estimations, this is where all the assigned developers will partake in "rounds" to agree on an estimate. Second is the "UCP Technique" for estimating the project using the use case points. This is traditional approach but will be provided to give the users more options. Third is the "Machine Learning" technique. This is the most advanced technique and has been tested to provide the best results in {\it{most}} cases. This freedom of choice will improve user experience by letting them choose the best approach for themselves.
\end{itemize}








\subsection{Executive Summary}
\blindtext


\blindtext

